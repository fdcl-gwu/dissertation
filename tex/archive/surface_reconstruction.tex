
Another approach is to use the sculpting of delauney triangulation~\cite{boissonnat1984}.
Further refinements came with the notion of \( \alpha \) shapes~\cite{edelsbrunner1994}.
The notion of a signed distance function and the implicit shape representation proposed~\cite{hoppe1992}.
A modification of this approach extended it to a voxel grid~\cite{curless1996}.
Ball pivoting allows for a relatively simple method of determining the surface~\cite{bernardini1999}.
Next, another method seeks to project sample points, with neighbors onto a plane then lift the local two dimensional delauney triangulation to reconstruct the surface~\cite{gopi2000}.

Our overarching goal for the surface reconstruction is to develop an efficient algorithm which allows for online, or near realtime, surface reconstruction from a sparse point cloud.
Furthermore, this method should allow for an incremental update given a new partial point cloud.
For example, given an initial coarse point cloud representing the initial guess of the surface, new, high density data, is acquired in a specific location on the surface.
Instead of processing the entire point cloud, and recomputing the global mesh, an incremental update would allow for modification of only the local neighborhood of the additional data.
Another useful requirment would be the ability to easily combine two overlapping point clouds into a single surface mesh.
Finally, the resulting surface should be topologically valid and without holes, i.e. a polyhedron mesh with triangular facets.



