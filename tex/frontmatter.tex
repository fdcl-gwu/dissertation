% !TEX root = ../dissertation.tex

% --------- FRONT MATTER PAGES ---------------------
% Title of the thesis
\title{Geometric Mechanics and Control for Small Body Missions}

% Author name
\author{Shankar Kulumani}

% Previous degrees
\bsdepartment{Astronautical Engineering}
\bsschool{United States Air Force Academy}
\bsgrad{May 2009}

\msdepartment{Aeronautical and Astronautical Engineering}
\msschool{Purdue University}
\msgrad{Dec 2013}
\showmsdegree % you can show or hide the MS degree line 
% \hidemsdegree

% PhD degree commands
% Committee
\showcommitteepage % hide this page if you're doing a MS thesis
\committee{ %
Taeyoung Lee Associate Professor of Engineering and Applied Science,\\ 
Dissertation Director\\ % remember to add a space between committee members

Michael Keidar A. James Clark Professor of Engineering, \\
Committee Member\\

Chung Hyuk Park Assistant Professor of Biomedical Engineering,\\
Committee Member \\

Islam Hussein, Senior Aerospace Research and Development Engineer, \\
Applied Defense Solutions, Committee Member\\

Tupper Hyde, Chief, Mission Engineering and Systems Analysis Division,\\
NASA Goddard Space Flight Center, Committee Member \\
}

% Chair must be entered separately for formatting reasons.
\chair{Taeyoung Lee}
\chairtitle{Associate Professor of Mechanical and Aerospace Engineering}
% Department
\department{Mechanical and Aerospace Engineering}

\phdgrad{August 31, 2018}
\defensedate{July 25, 2018}
% Year of completion for copyright page and perhaps other places
\year=2018

% Copyright page
\copyrightholder{Shankar Kulumani}

% Dedication
\dedication{ %
    \vfill
    \begin{center}For Christine\end{center}
    \vfill
}

% Acknowledgments
\acknowledgments{
    Jim Collins, in his book Good to Great, defined the ``hedgehog concept'' as a simple method to determine in which area one can be successful.
    The highly successful have found a profession in which they care deeply about, an area in which they can be the best, and one in which they can be well compensated. 
    I have never been the best in any endeavor and as any graduate student can attest, graduate school frequently does not make financial sense. 
    However, I am grateful for being able to do the things I truly want to do and this has made all the difference.

    The most valuable aspect of pursuing this work has been the complete freedom it has afforded me and that was only possible due to my adviser Professor Taeyoung Lee.
    Frequently, our interactions would involve me explaining how I had become sidetracked by yet another new project or completely unrelated hobby.
    He always provided the freedom to explore any idea or area, no matter how unrelated it may have seemed.
    It has been a privilege to receive his guidance, insight, and encouragement throughout my doctoral studies.
    His constant example taught me not only the technical skills required in this dissertation, but also how to live as a researcher.
    
    I would also like to thank all the members of the Flight Dynamics and Controls group for all the sunny days we got to spend inside behind a computer screen.
    Frequently, I would spend my time bothering the lab with my facts about cycling or movies they should watch. 
    In spite of this, everyone graciously accepted my annoyances and were always available to offer all the help I required.
    I enjoyed all the time we spent together and I hope I provided a fraction of the benefit that I gained from you.
    
    Finally, without the constant support of my wife, Christine, this dissertation would not have been possible. 
    She has contributed more to this work than any other and I am eternally grateful to be able to call her my wife.
    I love you with all my heart and always will.\nopagebreak
}

% -----------------------------------------------------------------
% Typically only one of Preface/Foreward/Prologue would be in your thesis.
% To choose one simply delete the others and they will automatically dissappear

% Preface
\preface{
    Preface
}

\prologue{
    Prologue
}

\foreword[2]{
    Foreword
}
% ----------------------------------------------------------------------

% commands to show or hide front matter pages
\showcopyright
\showabstract
\showcommitteepage
\showdedication
\showacknowledgments
\hidepreface
\hideprologue
\hideforeword
% ------------ TABLE OF CONTENTS ----------------------
% Commands to hide or show lists of figures, tables, etc.
\showlistoffigures
\showlistoftables
\hidenomenclature
\hidelistofabbreviations
\hidelistofsymbols

\setabbreviationstyle[acronym]{long-short}
\setabbreviationstyle[abbreviation]{long-short}
\makeglossaries
% you can hide/show the glossaries page
\showglossarieslistofabbreviations
\showglossarieslistofsymbols
\showglossariesglossaryofterms

% glossary  terms
\loadglsentries{tex/glossaries_terms.tex}
% abbreviations
\loadglsentries{tex/glossaries_abbreviations.tex}
% symbols
\loadglsentries{tex/glossaries_symbols.tex}
% Some abstract text
\abstract{

    This dissertation considers the operations of a rigid body spacecraft around a small body.
    Prior to arrival at a small body, such as an asteroid or comet, there is insufficient data to determine an accurate shape of the body, which is critical for the accurate computation of the gravitational acceleration. 
    As a result, the spacecraft will be entering into operation in a poorly modeled dynamic environment.
    Therefore, in order to operate autonomously the vehicle requires the following: a correct dynamic model to predict its own motion due to the gravitational effects of the small body, a control system to track a desired trajectory, and an efficient method to generate an estimate of the shape in order to determine the gravitational force.

    First, the complete dynamics of a rigid body spacecraft around a small body are derived and presented in four different forms.
    These equations of motion explicitly consider the interaction between the rotational and translational states of the vehicle.
    Furthermore, this dynamic model is specifically constructed on the nonlinear manifold of rigid body motions to correctly capture the geometric properties of the motion. 
    This geometric mechanics approach results in an intrinsic definition of the equations of motion that is globally valid and crucial for precise operations near a small body.

    Second, an optimal control formulation is presented for the low thrust orbital transfers of a spacecraft around an asteroid. 
    This approach enables the use of small magnitude propulsions systems to affect large scale orbital changes of the vehicle. 
    The optimal control problem is solved using the notion of a reachability set on a \Poincare map.
    This provides a representation of the maximum attainable states of a vehicle on a lower dimensional space.
    Additionally, a nonlinear geometric control system is developed for the closed loop control of the spacecraft.
    This control system enables precise trajectory tracking in the face of unknown disturbances or dynamic modeling errors. 

    Finally, a Bayesian framework is utilized for the shape reconstruction of an asteroid from range measurements. 
    This provides a computationally efficient method to incrementally update a small body shape. 
    The shape estimate is then utilized in an optimal guidance and control scheme to determine future states that best update the shape.
    A multi-resolution modeling approach is utilized to enable a higher fidelity shape representation in mission critical areas without a corresponding increase in the computational demand.
    These results are combined in a number of numerical simulations around asteroids.
    The results demonstrate the accurate shape reconstruction, precise guidance and control, and multi-resolution modeling that are required for spacecraft operations and landing at small bodies.
    
}
