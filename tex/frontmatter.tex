% !TEX root = ../dissertation.tex

% --------- FRONT MATTER PAGES ---------------------
% Title of the thesis
\title{Some kind of awesome title}

% Author name
\author{Shankar Kulumani}

% Previous degrees
\bsdepartment{Astronautical Engineering}
\bsschool{United States Air Force Academy}
\bsgrad{May 2009}

\msdepartment{Aeronautical and Astronautical Engineering}
\msschool{Purdue University}
\msgrad{Dec 2013}
\showmsdegree % you can show or hide the MS degree line 
% \hidemsdegree

% PhD degree commands
% Committee
% \showcommitteepage % hide this page if you're doing a MS thesis
\committee{ %
Taeyoung Lee, Associate Professor of Engineering and Applied Science,\\ 
Dissertation Director\\ % remember to add a space between committee members

Michael Keidar, A. James Clark Professor of Engineering, \\
Committee Member\\

Chung Hyuk Park, Assistant Professor,\\
Committee Member
}

% Chair must be entered separately for formatting reasons.
\chair{Taeyoung Lee}
\chairtitle{Associate Professor of Mechanical and Aerospace Engineering}
% Department
\department{Mechanical and Aerospace Engineering}

\phdgrad{May 28, 2018}
\defensedate{May 28, 2018}
% Year of completion for copyright page and perhaps other places
\year=2018

% Copyright page
%\copyrightholder{Someone else}

% Dedication
\dedication[8]{ %
\textit{Some cool dedication}
}

% Acknowledgments
\acknowledgments{
    More cool acknowledgements of allthe people who have helped
    This research has been supported in part by NSF under the grants CMMI-1243000, CMMI-1335008, and CNS-1337722.
}

% -----------------------------------------------------------------
% Typically only one of Preface/Foreward/Prologue would be in your thesis.
% To choose one simply delete the others and they will automatically dissappear

% Preface
\preface{
    Preface
}

\prologue{
    Prologue
}

\foreword[2]{
    Foreword
}
% ----------------------------------------------------------------------

% commands to show or hide front matter pages

\hidecopyright
\hideabstract
\hidecommitteepage
\hidededication
\hideacknowledgments
\hidepreface
\hideprologue
\hideforeword
% ------------ TABLE OF CONTENTS ----------------------
% Commands to hide or show lists of figures, tables, etc.
\showlistoffigures
\showlistoftables
\hidenomenclature
\hidelistofabbreviations
\hidelistofsymbols

\setabbreviationstyle[acronym]{long-short}
\setabbreviationstyle[abbreviation]{long-short}
\makeglossaries
% you can hide/show the glossaries page
\showglossarieslistofabbreviations
\showglossarieslistofsymbols
\showglossariesglossaryofterms

% TODO Look up how to manage these definitions using a seperate file (bib etc)
% abbreviations
% \newabbreviation{crtbp}{CRTBP}{Circular Restricted Three Body Problem}
% defining abbreviations like this allows for autocompletion
\newglossaryentry{filo}{
    name={FILO},
    type=\glsxtrabbrvtype,
    description={first in last out},
    first={first in last out (FILO)}
}

\newglossaryentry{tof}{
    name={TOF},
    type=\glsxtrabbrvtype,
    description={time of flight},
    first={time of flight (TOF)}
}

\newglossaryentry{lidar}{
    name={LIDAR},
    type=\glsxtrabbrvtype,
    description={Light Detection and Ranging},
    first={Light Detection and Ranging (LIDAR)}
}

\newglossaryentry{crtbp}{
    name={CRTBP},
    type=\glsxtrabbrvtype,
    description={Circular Restricted Three Body Problem},
    first={Circular Restricted Three Body Problem (CRTBP)}
}

\newglossaryentry{gis}{
    name={GIS},
    type=\glsxtrabbrvtype,
    description={Geographics Information System},
    first={Geographic Information System (GIS)}
}

% glossary  terms
\newglossaryentry{linux}{
    name=Linux,
    description={is a generic term referring to the family of Unix-like computer operating systems that use the Linux kernel},
    plural=Linuces
}

\newglossaryentry{matrix}{
    name={matrix},
    plural={matrices},
    description={rectangular array of quanttities}
}

\newglossaryentry{frustum}{
    name={frustum},
    plural={frusta},
    description={In geometry: defined as the portion of a solid, normally a cone or pyramid, that lies between two parallel planes.
    In computer graphics: the viewing frustum is the three-dimensional region which is visible on screen.}
}

\newglossaryentry{point_cloud}{
    name={point cloud},
    plural={point clouds},
    description={A set of points in some reference system.
        In a three dimension corrdination system, the points typical represent the cartesian position of the external surface of an object.
    Frequently generated from a 3D scanning process, these points are rarely used directly but rather converted into a surface representation, such as a polygonal mesh.}
}

\newglossaryentry{implicit_shape}{
    name={implicit shape},
    plural={implicit shapes},
    description={In mathematics: an implicit shape or surface is the surface in Euclidean space defined by an equation \( F(x, y, z) = 0\).
    The implicit surface is the zero set of the function \( F \), and is termed \textit{implicit} as the function is not solved for \( x, y, \text{ or } z \).}
}

\newglossaryentry{polygon}{
    name={polygon},
    plural={polygons},
    description={The area of a plane bounded by straight segments, called edges, which intersect only at their endpoints, called vertices.
    Each vertex belongs to exactly two edges and similarly each edge has exactly two vertices.}
}

\newglossaryentry{MAT}{
    name={medial axis transform},
    description={The medial axis is the set of all point shaving more than one closest point on the objects boundary.
    The medial axis along with the associated radius function of the maximally inscribed discs is called the medial axis transform.}
}
% symbols
\newglossaryentry{M}{
    type=symbols,
    name={\ensuremath{M}},
    sort=M,
    description={a \gls{matrix}}
}

\newglossaryentry{F}{
    type=symbols,
    name={\ensuremath{F}},
    sort=F,
    description={External Force}
}

\newglossaryentry{rho}{
    type=symbols,
    name={\ensuremath{\vec \rho}},
    sort=R,
    description={Range to target}
}

\newglossaryentry{view_axis}{
    type=symbols,
    name={\ensuremath{\vec d}},
    sort=D,
    description={Sensor optical axis}
}

\newglossaryentry{view_distance}{
    type=symbols,
    name={\ensuremath{d}},
    sort=D,
    description={Distance from the sensor, along the view axis, to the far plane defining the sensor viewing frustum},
    see={frustum,view_axis}
}

\newglossaryentry{up_axis}{
    type=symbols,
    name={\ensuremath{\vec u}},
    sort=U,
    description={Sensor up axis}
}

\newglossaryentry{fov_h}{
    type=symbols,
    name={\ensuremath{\alpha}},
    sort=A,
    description={Horizontal sensor field of view}
}

\newglossaryentry{fov_v}{
    type=symbols,
    name={\ensuremath{\beta}},
    sort=B,
    description={Vertical sensor field of view}
}

% Some abstract text
\abstract{
    Abstract
}
