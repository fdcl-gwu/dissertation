% !TEX root = ../dissertation.tex

\chapter{Nonlinear Geometric Control}\label{sec:se3_control}

While there has been significant study of interplanetary transfer trajectories, relatively less analysis has been conducted on operations in the vicinity of asteroids.
The dynamic environment around asteroids is strongly perturbed and challenging for analysis and mission operations~\cite{scheeres1994,scheeres2000}.
Due to their low mass, which results in a low gravitational attraction, asteroids may have irregular shapes and potentially chaotic spin states.
As a result, typical approaches of assuming a inverse square gravitational model are at best inaccurate and at worst do not capture the true dynamic environment.
In addition, the vast majority of asteroid are difficult to track or measure using current ground-based optical sensors. 
Due to their small size, frequently less than \SI{1}{\kilo\meter}, and low albedo the reflected energy of these asteroids is insufficient for reliable detection or tracking.
Therefore, the dynamic model of the asteroid is relatively coarse prior to arrival of a dedicated spacecraft in the vicinity. 
As a result, any spacecraft mission to an asteroid is dependent on a robust dynamic simulation and must incorporate the ability to deal with uncertain forces and environments.

% talk about the coupling between attitude and translational states
Furthermore, since the magnitude of the gravitational attraction is relatively small, non-gravitational effects, such as solar radiation pressure or third-body effects, become much more significant.
As a result, the orbital environment is generally quite complex and it is difficult to generate analytical insights.
One key consideration is the coupling between rotational and translational states around the asteroid.
The coupling is induced due to the different gravitational forces experienced on various parts of the spacecraft.
The effect of the gravitational coupling is related to the parameter \(\epsilon = \frac{r}{R_c}\), where \(r\) is the characteristic spacecraft length and \(R_c\) is the orbital radius~\cite{hughes2004}.
For Earth based missions, the orbital radius is several orders of magnitude larger than the spacecraft length and \(\epsilon\) is small.
As a result, the corresponding gravitational moment is weak and can be neglected. 
Therefore, the translational and rotational equations of motion become decoupled and can be considered separately, significantly simplifying the analysis. 
However, for operations around an asteroid the orbital radius is much smaller, which leads to much larger values of \(\epsilon\) and much larger influence of the rotational and translational coupling.
References~\cite{elmasri2005} and~\cite{sanyal2004} investigated the coupling of an elastic dumbbell spacecraft in orbit about a central body, but only considered the case of a spherically symmetric central body.
Furthermore, the spacecraft model is assumed to remain in a planar orbit.
As result, these developments are not directly applicable to motion about an asteroid, which experience highly non-keplerian dynamics.

% now talk about the specific challenges of asteroid landing. there are lots of trajectory design papers and several have considered landing trajectories
An additional layer of complexity is the design of landing trajectories on asteroids.
Beginning with the first landing of NEAR Shoemaker on asteroid 433 Eros, there has been a concerted effort to develop techniques and methodologies for asteroid landing~\cite{dunham2002, kubota2006}.
There is already considerable knowledge on the planetary landing problem~\cite{acikmese2007, meditch1964, ingoldby1978}.
While conceptually similar, the landing of spacecraft on small bodies requires additional consideration. 
The surface of an asteroid is highly irregular and, as discussed previously, there is a large coupling between the translational and rotational dynamics of the vehicle, which is further exaggerated when close to the surface.
References~\cite{guelman1994, furfaro2013, zexu2012} consider the soft landing problem on an asteroid.
These approaches were primarily based on nonlinear control techniques which allowed for the development of closed loop controllers which enable landing.
However, only the translational dynamics of the body was considered and no notion of the attitude dynamics or it's coupling to the position is considered.
Furthermore, relatively simple gravitational models are used which make the results unsuitable for operations near irregular bodies.
In this chapter, we derive and develop a nonlinear geometric controller for the coupled dynamics of a spacecraft around an asteroid.

An additional layer of complexity is the design of landing trajectories on asteroids.
Beginning with the first landing of NEAR Shoemaker on asteroid 433 Eros, there has been a concerted effort to develop techniques and methodologies for asteroid landing~\cite{dunham2002, kubota2006}.
There is already considerable knowledge on the planetary landing problem~\cite{acikmese2007, meditch1964, ingoldby1978}.
While conceptually similar, the landing of spacecraft on small bodies requires additional consideration. 
The surface of an asteroid is highly irregular and, as discussed previously, there is a large coupling between the translational and rotational dynamics of the vehicle, which is further exaggerated when close to the surface.
References~\cite{guelman1994, furfaro2013, zexu2012} consider the soft landing problem on an asteroid.
These approaches were primarily based on nonlinear control techniques which allowed for the development of closed loop controllers which enable landing.
However, only the translational dynamics of the body was considered and no notion of the attitude dynamics or it's coupling to the position is considered.
Furthermore, relatively simple gravitational models are used which make the results unsuitable for operations near irregular bodies.
\section{Geometric Control}

A wide variety of control schemes have been proposed for asteroid landing missions~\cite{furfaro2013,li2011a}.
In addition, there are a  variety of controllers developed for systems evolving on \( \SE \)~\cite{lee2010,lee2013}.
In this paper, we extend their use from quadrotor aerial vehicles into the space domain. 
This approach addresses many of the issues associated with the related work on asteroid landings.
The geometric control methods used to develop these nonlinear controllers allow for the development of control systems for dynamic systems which evolve on nonlinear manifolds. 
By developing the control system directly on the nonlinear manifold, geometric control techniques provide unique advantages as compared to those developed using local coordinate representations.
Furthermore, the geometric controller avoids the chattering issues inherent in the previous sliding mode control approaches to asteroid landing.
In addition, rather than offering only a bounded stability guarantee, the proposed nonlinear geometric controller guarantees almost global tracking of the attitude and translational states. 
This stability guarantee is critical for mission operations passing close to the surface over highly irregular terrain.
Furthermore, the coupled geometric controller explicitly considers the attitude coupling of the body in contrast to many of the previous approaches.
We briefly summarize the key developments of the \( \SE \) control scheme and leave the detailed derivations to the source manuscripts~\cite{lee2010,lee2013}.
% TODO Basics of geometric control

% TODO Derive attitude and translation control seperately

% TODO Specifics for use in the asteroid case

\subsection{Attitude Control}
In order to determine the attitude control input, we first define a desired attitude tracking command.
An arbitrary smooth attitude tracking command \( R_d (t) \in \SO \) is given as a function of time.
The corresponding angular velocity command is obtained using the attitude kinematics equation, \( \hat{\Omega}_d = R_d^T \dot{R}_d \).
With the desired attitude command, we then define the errors associated with the attitude and angular velocity.
The attitude and angular velocity tracking errors must be careful chosen to remain on the tangent bundle of \SO.
First, an attitude error function is defined on \( \SO \times \SO \) as
\begin{align}\label{eq:attitude_error_function}
    \Psi(R, R_d) = \frac{1}{2} \tr{I - R_d^T R}.
\end{align}
This positive definite function parameterizes the error between the current attitude, \( R \), and the desired attitude command \( R_d \).
Using the variations of \( \Psi \) gives the attitude tracking error vector \( e_R \in \R^3 \) as
\begin{align}\label{eq:attitude_error_vector}
    e_R = \frac{1}{2} \parenth{R_d^T R - R^T R_d^\vee}.
\end{align}
After further manipulation and using the attitude kinematics equation from~\cref{eq:attitude_kinematics}, it is possible to define the angular veloicty tracking error \( e_\Omega \in \R^3 \) as
\begin{align}\label{eq:angular_velocity_error_vector}
    e_\Omega = \Omega - R^T R_d \Omega_d .
\end{align}
With the properly defined attitude error vectors the rotational control input is defined as 
\begin{align*}\label{eq:rotational_control}
    \vb{u}_m = - k_R e_R - k_\Omega e_\Omega + \Omega \times J \Omega - J \parenth{\hat{\Omega} R^T R_d \Omega_d - R^T R_d \dot{\Omega}_d} - \vb{M}_1 - \vb{M_2} 
\end{align*}
where \( k_R, k_\Omega \) are positive controller constants.

\subsection{Translation Control}
The translational control input is defined in a similar manner. 
First we define a smooth tracking command \( x_d(t) \in \R^3 \), which defines the desired position of the spacecraft in the inertial frame.
The tracking error vectors are easier to define as they evolve on a Euclidean space rather than a nonlinear manifold and are given by
\begin{align}
    e_x = x - x_d ,\\
    e_v = v - \dot{x}_d .
\end{align}
With the error variables, the translational control input is then given by
\begin{align}\label{eq:translation_control}
    \vb{u}_f = - k_x e_x  - k_v e_v + ( m_1  + m_2 ) \ddot{x}_d - \vb{F}_1 - \vb{F}_2 ,
\end{align}
where \( k_x, k_v \) are positive constants. 
The control gains are chosen based on the desired closed-loop system response. 
A variety of techniques are available to choose these gains, but a simple linear analysis offers a straightforward and systematic approach to choosing suitable values. 
We use the control inputs defined in~\cref{eq:translation_control,eq:rotational_control} and substitute them into the dynamic equations of motion in~\cref{eq:translational_dynamics,eq:attitude_dynamics}.
This results in the dynamics of the error variables and the gains are chosen to ensure the error behavior meets desired performance criteria, such as percent overshoot or settling time~\cite{nise2004}.


\section{Numerical Example}

% TODO Disucss numerical implementation
