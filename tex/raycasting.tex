% !TEX root = ../dissertation.tex

\chapter{Determining the shape of an asteroid}

\section{Raycasting}

In the computer graphics field a major problem is the determination of which objects are visible to the user.
In a method similar to the photographic process in reverse, for each pixel of the image a ray is cast towards the scene and the intersections of this ray with any objects is recorded.
The first object which intersects the ray is defined as visible while those beyond would be occluded.
Given this intersection point, the surface normal is computed and surface shading can be determined to render the scene.

Our problem is of a similar nature to that of the computer graphics domain.
Given the model of an asteroid, its shape model, we wish to first simulate measurements of the surface. 
In order to simulate these methods, we must also implement a ray casting algorithm that will find the intersection of a measurement from the spacecraft to the surface. 

We assume that the spacecraft is located at \gls{rho} in the asteroid fixed frame.
The camera/sensor is aligned with \gls{view_axis} in the spacecraft fixed frame.
An additional vector, defined as the up axis \gls{up_axis}, finally the standard basis is completed with \( \gls{view_axis} \times \gls{up_axis} \) which defines a reference frame attached to the image plane of the sensor.
Furthermore, we model the the laser ranging sensor as having a fixed field of view defined by the angles \( \gls{fov_h}, \gls{fov_v}\), defining the vertical and horizontal fields of view.
In order to simulate depth measurements we need to compute the vectors associated with image sensor in the asteroid frame. 
Given the view axis, and a chosen distance \gls{view_distance}, we can define a viewing \gls{frustum} associated with the sensor. 
We can compute the vectors associated with the maximum extents of the far plane as follows.
The half height and width of the far plane, at a distance \( d \), is computed as
\begin{align}
    H = d \tan \frac{\alpha}{2} , \\
    W = d \tan \frac{\beta}{2} .
\end{align}
From these values, the extents of the far plane are defined by the vectors
\begin{align}
    A = \hat d + H \hat u - W \hat r , \\
    B = \hat d + H \hat u + W \hat r, \\
    C = \hat d - H \hat  u + W \hat r, \\
    D = \hat d - H \hat u - W \hat r.
\end{align}

% TODO: Explain more about ray casting

% TODO: Talk about laser range finders on spacecraft

\section{Laser Range Finder on Spacecraft}

Absolute distance measuring devices are a crucial sensor in spacecraft rendezvous or landing applications.
In addition, they are a critical component for spacecraft operating near asteroids.
Activate ranging devices allow for precise measurments of the surface shape and enable accurate and safe surface landing~\cite{berry2013}.

The basic principle central to all active range-finding devices is to transmit a signal onto an object and process the returned signal to determine the distance~\cite{amann2001}.
The transmitted signal can fall into one of three categories: radio, ultrasonic, or optical.
The benefit of optical signal is that they can be highly focused to enable high resolution distance measurments
Optical distance measurment, such as laser range finders, can be further divided into three categories based on the method of operation: interferometry, \gls{tof}, or triangulation.
In this section, we'll briefly summarize the principles of operation of optical distance measurement devices, and highlight their specific uses with respect to spacecraft missions to asteroids.

\subsection{Time of Flight Distance Measurement}
Originally developed for military and surveying application, the basic principle of laser ranging is based on utilizing the fixed speed of light to measure distance.
The time required for a pulse of energy to travel from the transmitter to the object and return is measured, \( t_d \). 
Meausuring this round trip \ac{TOF} and using the speed of light, approximately \( c = \SI{30}{\centi\meter\per\nano\second}\), allows one to easily compute the range to the object as
\begin{align}
    \rho = \frac{t_d}{2 c}
\end{align}
% TODO Discuss how the shape of an asteroid is determined in pracice currently

% TODO Talk about how we construct a mesh given a point cloud


