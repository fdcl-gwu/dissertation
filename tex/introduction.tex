% !TEX root = ../dissertation.tex

\chapter{Introduction}

\section{Title}
\begin{itemize}
    \item Precise set of words that people will eventually google
    \item No more than 7-8 words. The keywords from teh abstract
\end{itemize}

\begin{itemize}
    \item Motivation the problem and state my hypothesis
        \begin{itemize}
            \item Tell a story
            \item Use concrete examples (or a running example) and lots of figures
            \item Quote data sources, case studies or real missions
            \item Introduce all the terminology
            \item Why this work is important
            \item What makes it special and good
        \end{itemize}
    \item Literature review and past contributions
        \begin{itemize}
            \item Talk about previous research.
            \item Try to highlight areas where topics were lacking, that my work will fill
        \end{itemize}
    \item Current contributions
        \begin{itemize}
            \item Explicit contributions of the current disseration
            \item State all \emph{assumptions} and \emph{limitations}
            \item All the \emph{requirements}/\emph{constraints} of this approach
            \item Concrete \emph{validation} of the plan, experiments/simulation
            \item Define the \emph{scope}
        \end{itemize}
    \item Dissertation Overview
        \begin{itemize}
            \item Introduce each subsequent section
            \item Talk about how they all fit together
            \item Sequence of 2-3 distinct top conference papers
            \item Natural progression from each chapter (conference paper)
        \end{itemize}

        
\end{itemize}

\section{Prensenting the thesis}
\begin{itemize}
    \item Explain why I picked this approach
    \item Philosophy/thought process that lead to this approach
    \item What other approaches were considered and discarded? Why
    \item \emph{How did I know I was done?}
    \item \emph{When is the problem solved?}
    \item \emph{If you had to do it all over again would you do it differently?}
\end{itemize}

\begin{enumerate}
    \item Introduction
        \begin{itemize}
            \item Overall goal is to enable autonomous mapping and control about an asteroid using LIDAR
        \end{itemize}
    \item Mathematical background
        \begin{itemize}
            \item Three body dynamics
            \item Motion around an asteroid
            \item Polyhedron potential models
            \item Variational Integrators
            \item Geometric control
            \item Rigid body dynamics
        \end{itemize}
    \item Orbital transfers using reachability
    \item Geometric control for coupled motion
    \item Asteroid shape estimation and generation
    \item End-to-end mission examples
    \item Conclusions
        \begin{itemize}
            \item Areas for future research
        \end{itemize}
\end{enumerate}


\section{Each thesis chapter}
\begin{itemize}
    \item Introduction
        \begin{itemize}
            \item What is the chapter about?
            \item What sub-problem/issue is this chapter addressing?
            \item How does chapter fit within the overall story of the thesis?
        \end{itemize}
    \item The meat of the chapter
        \begin{itemize}
            \item Rigorous approach to the sub-problem or detailed explanation of issue
            \item Assumptions underlying this subproblem
            \item Validation: design, theory, implementation, graphs, figures
        \end{itemize}

    \item Summary
        \begin{itemize}
            \item Repeat the highlights of the chapter
            \item Transition to the next chapter. 
                How the chapter fits with the next one.
        \end{itemize}
\end{itemize}

\section{Summary/Slides}
\begin{itemize}
    \item Problem statement/hypothesis
    \item Why this is a hard/difficult problem?
    \item Approach/Methodology
    \item Why is this innovative/new?
    \item Assumptions/constraints?
    \item Initial results
    \item  Validation plan
    \item Limitations and applicability
    \item Expected contributions
    \item What's beyond  this thesis?
    \item Roadmap for the thesis
\end{itemize}

be able to answer the following questions
\begin{itemize}
    \item Why will work change the world?
    \item What would be done differently if I could start over?
    \item How will you know when success is reached?
    \item How much of the work can be generalized?
    \item Which part was research and which was engineering?
\end{itemize}


\section{Computational Geometry}

This section can discuss some background in computational geometry.

The majority of computational geometry algorithms are developed for simple planar \glspl{polygon}.

The surface of an object is described as the union of the surfaces of the polygons which define it.


\subsection{Surface Reconstruction}
One of the main mission objectives for asteroid operations is to generate the three-dimensional shape of an asteroid.
This shape generation is a subset of a larger class of methods called surface reconstruction.
The goal of surface reconstruction is stated as follows: Given a set of sample points \( X\) assumed to lie on /near the surface \( U \), create a surface model \( S \) approximating \( U \).
Surface reconstruction is a well-researched field and widely applied in the industrial and computer vision fields.
The generation of a surface from an object is important in reverse engineering, product design and medical product construction~\cite{amenta2001}.
The surface reconstruction problem is applied in a variety of fields in science and engineering including 3D scanning, surface reconstruction from contours and surface sketching~\cite{hoppe1992}.

Surface reconstruction, or sometimes referred to as 3D scanning, is the generation of accurate models of real world objects.
Surface Reconstruction techniques are frequently used to digitize sculptures/architecture, machine parts, archaeological artifacts, or terrain.
There are a variety of methods to digitize objects, such as laser range finding, mechanical touch probes, or computer vision techniques, i.e. depth from stereo.
From these digitization methods, a large collection, often times on the order of \( 10^6 - 10^9\), of points are generated.
From this \gls{point_cloud}, the goal is to construct a surface representation which is faithful to the original data, and the underlying object of interest.
This process of generating the surface should ideally have the following properties:
\begin{itemize}
    \item Fast - low computational cost
    \item Low memory - storing/operating on large collection of points is frequently memory constrained. Also for use in constrained enviornments
    \item Robustness - Algorithm must be robust to holes, noisy data, varying or low sample density
    \item Accuracy - the method should approximate the shape based on the data rather than interpolating between data points
    \item Features - ability to capture sharp edges or features
\end{itemize}

% TODO Describe different methods of surface reconstruction

The \gls{implicit_shape} representation allows for a complex shape to be defined by a single equation.
This approach has the additional benefit of unifying both the surface and volume rendering to a single equation.
The main advantage associated with implicit surface approach is handling sparse or scattered data.
It is possible to repair or edit the shape using standard implicit modeling operations.
For example, given two implicit surfaces
\begin{align*}
    F_1(x, y, z) &= 0, \\
    F_2(x, y, z) &= 0 ,
\end{align*}
one can define a new surface as
\begin{align*}
    F(x, y, z) = \mu F_1(x, y, z) + (1-\mu) F_2(x,y,z) = 0,
\end{align*}
where \( \mu \in \bracket{0, 1}\) is a design parameters defining the blending between the two input surfaces.
This blending allows one to use primitive shapes/models to create complex surfaces.
% TODO Find citation for this (blending of implicit shapes)
However, this implicit representation makes it difficult to modify the local shape of the surface. 
Examples of this approach include~\cite{bolitho2009,hoppe1992,ohtake2005}.
% TODO Describe in more detail the difference, drawbacks/advantages, of each implicit surface approach.

Surface reconstruction was first addressed in the early 80's~\cite{uselton1983,uselton1981}.
The main goal of this early work was focused on developing methods to accurately and efficiently display three dimensional data on a computing system.
The resulting work developed a computational algorithm to construct the three-dimensional polygonal representation of an object from the collection of points describing the surface.
The determination of a polygonal surface through a series of points is a relatively straightforward problem, as there an infinite number of solutions.
However, determining the ``best'' surface is a more challenging problem.
As the original surface is typically not available, hence the need for measurement and digitization, the quality of the resulting surface must be computed using the measurement points.

Another approach is to use the sculpting of delauney triangulation~\cite{boissonnat1984}.
Further refinements came with the notion of \( \alpha \) shapes~\cite{edelsbrunner1994}.
The notion of a signed distance function and the implicit shape representation proposed~\cite{hoppe1992}.
A modification of this approach extended it to a voxel grid~\cite{curless1996}.
Ball pivoting allows for a relatively simple method of determining the surface~\cite{bernardini1999}.
Next, another method seeks to project sample points, with neighbors onto a plane then lift the local two dimensional delauney triangulation to reconstruct the surface~\cite{gopi2000}.

Our overarching goal for the surface reconstruction is to develop an efficient algorithm which allows for online, or near realtime, surface reconstruction from a sparse point cloud.
Furthermore, this method should allow for an incremental update given a new partial point cloud.
For example, given an initial coarse point cloud representing the initial guess of the surface, new, high density data, is acquired in a specific location on the surface.
Instead of processing the entire point cloud, and recomputing the global mesh, an incremental update would allow for modification of only the local neighborhood of the additional data.
Another useful requirment would be the ability to easily combine two overlapping point clouds into a single surface mesh.
Finally, the resulting surface should be topologically valid and without holes, i.e. a polyhedron mesh with triangular facets.

\subsubsection{Watertight surface mesh}

There are two main approaches to generate water tight surface meshes.
The powercrust algorithm seeks to approximate the \gls{MAT} of a three-dimensional object~\cite{amenta2001}.
Then the inverse map is used to produce a watertight boundary.

Another approach is the tight cocone algorithm~\cite{dey2006}.
This reduced the complexity of powercrust and uses a post processing method to fill any holes.

\subsubsection{3D Scanning}
There are several different approches to acquire shape information of an object.
The computer vision field several approaches exist to determine the shape of an object.
In feature correspondance, landmarks are registered and compared across several different  views of an object. 
The shape of the surface is inferred given the known motion of the camera system~\cite{szeliski2010}.
Another approach is based on the different surface reflectivity of the surface, called shape from shading~\cite{szeliski2010}.

% TODO Mechanical scanning using a probe or laser scanner 
% TODO Range images are generated with distances to the surface. Assume known motion of camera (SLAM) if not known

In 
% TODO Talk about how we construct a mesh given a point cloud
\begin{itemize}
    \item Review surface reconstruction algorithms
    \item Talk about the one that is implemented inside VTK Hoppe papers
\end{itemize}
