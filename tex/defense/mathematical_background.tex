\begin{frame}{Reference Frames}
   
    \begin{columns}
        \begin{column}{0.5\textwidth}
    \begin{enumerate}
        \item \( \vc{e}_i \in \R^3 \) : inertial reference frame
        \item \( \vc{f}_i \in \R^3 \) : small body fixed frame
        \item \( \vc{b}_i \in \R^3 \) : spacecraft fixed frame
    \end{enumerate}
\end{column}
\begin{column}{0.5\textwidth}
    \begin{scaletikzpicturetowidth}{2\columnwidth}
        \tdplotsetmaincoords{60}{110}
        \begin{tikzpicture}[scale=\tikzscale,tdplot_main_coords]
            %define polar coordinates for some vector
            \pgfmathsetmacro{\rvec}{.8}
            \pgfmathsetmacro{\thetavec}{30}
            \pgfmathsetmacro{\phivec}{60}
            \pgfmathsetmacro{\Px}{0.2}
            \pgfmathsetmacro{\Py}{0.346}
            \pgfmathsetmacro{\Pz}{0.693}
            \pgfmathsetmacro{\b}{0.5}
            \pgfmathsetmacro{\d}{0.3}

            %set up some coordinates 
            %-----------------------
            \coordinate (O) at (0,0,0);
            \coordinate (Ptikz) at (\Px, \Py, \Pz);
            %determine a coordinate (P) using (r,\theta,\phi) coordinates.  This command
            %also determines (Pxy), (Pxz), and (Pyz): the xy-, xz-, and yz-projections
            %of the point (P).
            %syntax: \tdplotsetcoord{Coordinate name without parentheses}{r}{\theta}{\phi}
            % \tdplotsetcoord{P}{\rvec}{\thetavec}{\phivec}
            %draw figure contents
            %--------------------

            %draw the main coordinate system axes
            \draw[thick,-Latex] (0,0,0) -- (1,0,0) node[anchor=north east]{$e_1$};
            \draw[thick,-Latex] (0,0,0) -- (0,1,0) node[anchor=north west]{$e_2$};
            \draw[thick,-Latex] (0,0,0) -- (0,0,1) node[anchor=south]{$e_3 = f_3$};

            %draw a vector from origin to point (P) 
            \draw[-stealth,color=red, -Latex] (O) -- node [below right] {$ r $} ($(\Px, \Py, \Pz) $);

            %draw the angle \phi, and label it
            %syntax: \tdplotdrawarc[coordinate frame, draw options]{center point}{r}{angle}{label options}{label}
            \tdplotdrawarc{(O)}{0.2}{0}{\phivec}{anchor=north}{$\Omega_A t$}


            %set the rotated coordinate system so the x'-y' plane lies within the
            %"theta plane" of the main coordinate system
            %syntax: \tdplotsetthetaplanecoords{\phi}
            \tdplotsetthetaplanecoords{\phivec}

            % draw the asteroid frame
            % \draw[thick,->,tdplot_rotated_coords] (0,0,0) -- (1,0,0) node[anchor=north east]{$f_3$};
            \draw[thick,-Latex,tdplot_rotated_coords] (0,0,0) -- (0,1,0) node[anchor=north west]{$f_1$};
            % \draw[thick,->,tdplot_rotated_coords] (0,0,0) -- (0,0,1) node[anchor=north west]{$f_2$};

            %draw some dashed arcs, demonstrating direct arc drawing
            \draw[dashed,tdplot_rotated_coords] (\rvec,0,0) arc (0:90:\rvec);
            \draw[dashed] (\rvec,0,0) arc (0:90:\rvec);

            %set the rotated coordinate definition within display using a translation
            %coordinate and Euler angles in the "z(\alpha)y(\beta)z(\gamma)" euler rotation convention
            %syntax: \tdplotsetrotatedcoords{\alpha}{\beta}{\gamma}
            \tdplotsetrotatedcoords{\phivec}{\thetavec}{0}

            %translate the rotated coordinate system
            %syntax: \tdplotsetrotatedcoordsorigin{point}
            \tdplotsetrotatedcoordsorigin{(Ptikz)}

            %use the tdplot_rotated_coords style to work in the rotated, translated coordinate frame
            \draw[thick,tdplot_rotated_coords,-Latex] (0,0,0) -- (\b,0,0) node[anchor=north west]{$b_1$};
            \draw[thick,tdplot_rotated_coords,-Latex] (0,0,0) -- (0,\b,0) node[anchor=west]{$b_2$};
            \draw[thick,tdplot_rotated_coords,-Latex] (0,0,0) -- (0,0,\b) node[anchor=south]{$b_3$};

            % transform the first dumbbell
            % rot to main
            \tdplottransformrotmain{\d}{0}{0};
            \tdplottransformmainscreen{\tdplotresx}{\tdplotresy}{\tdplotresz};
            \pgfmathsetmacro{\DMonescreenx}{\tdplotresx};
            \pgfmathsetmacro{\DMonescreeny}{\tdplotresy};

            \tdplottransformrotmain{-\d}{0}{0};
            \tdplottransformmainscreen{\tdplotresx}{\tdplotresy}{\tdplotresz};
            \pgfmathsetmacro{\DMtwoscreenx}{\tdplotresx};
            \pgfmathsetmacro{\DMtwoscreeny}{\tdplotresy};

            % transform P to screen coordinates and then add
            \tdplottransformmainscreen{\Px}{\Py}{\Pz};
            \pgfmathsetmacro{\Pscreenx}{\tdplotresx};
            \pgfmathsetmacro{\Pscreeny}{\tdplotresy};

            \draw[ultra thick, color=blue, tdplot_rotated_coords] (-\d, 0, 0) -- (\d, 0, 0);

            \shade[ball color=red, tdplot_screen_coords, opacity=0.3] (\Pscreenx, \Pscreeny) circle (0.03);
            \shade[ball color=blue,tdplot_screen_coords] ($(\Pscreenx, \Pscreeny) + (\DMonescreenx, \DMonescreeny) + (0, 0.015)$) circle (0.03);
            \shade[ball color=blue,tdplot_screen_coords] ($(\Pscreenx, \Pscreeny) + (\DMtwoscreenx, \DMtwoscreeny) + (0, -0.015)$ ) circle (0.03);

        \end{tikzpicture}
    \end{scaletikzpicturetowidth}
\end{column}
\end{columns}
\end{frame}
