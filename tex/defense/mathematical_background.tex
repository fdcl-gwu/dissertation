% !TEX root = ../../defense.tex

\section{System Model}

\begin{frame}{Reference Frames}

    \begin{columns}
        \begin{column}{0.5\textwidth}
            \begin{enumerate}
                \item \( \vc{e}_i \in \R^3 \) : inertial reference frame fixed in space
                \item \( \vc{f}_i \in \R^3 \) : small body fixed frame aligned with principle axes of the body
                \item \( \vc{b}_i \in \R^3 \) : spacecraft fixed frame with \( b_1 \) aligned with the symmetry axis
            \end{enumerate}
            
            \begin{block}{}
                Spacecraft state is defined on the Special Euclidean group

                \[ \parenth{R, \vc{x}} \in \SE \]
            \end{block}
        \end{column}
        \begin{column}{0.5\textwidth}
            \begin{scaletikzpicturetowidth}{2\columnwidth}
                \resizebox{\columnwidth}{!}{%
                %set the plot display orientation
%synatax: \tdplotsetdisplay{\theta_d}{\phi_d}
\tikzsetnextfilename{ref_frames}
\tdplotsetmaincoords{60}{110}
%start tikz picture, and use the tdplot_main_coords style to implement the display 
%coordinate transformation provided by 3dplot
\begin{tikzpicture}[scale=5,tdplot_main_coords]
%define polar coordinates for some vector
\pgfmathsetmacro{\rvec}{.8}
\pgfmathsetmacro{\thetavec}{30}
\pgfmathsetmacro{\phivec}{60}
\pgfmathsetmacro{\Px}{0.2}
\pgfmathsetmacro{\Py}{0.346}
\pgfmathsetmacro{\Pz}{0.693}
\pgfmathsetmacro{\b}{0.5}
\pgfmathsetmacro{\d}{0.3}

%set up some coordinates 
%-----------------------
\coordinate (O) at (0,0,0);
\coordinate (Ptikz) at (\Px, \Py, \Pz);
%determine a coordinate (P) using (r,\theta,\phi) coordinates.  This command
%also determines (Pxy), (Pxz), and (Pyz): the xy-, xz-, and yz-projections
%of the point (P).
%syntax: \tdplotsetcoord{Coordinate name without parentheses}{r}{\theta}{\phi}
% \tdplotsetcoord{P}{\rvec}{\thetavec}{\phivec}
%draw figure contents
%--------------------

%draw the main coordinate system axes
\draw[thick,-Latex] (0,0,0) -- (1,0,0) node[anchor=north east]{$e_1$};
\draw[thick,-Latex] (0,0,0) -- (0,1,0) node[anchor=north west]{$e_2$};
\draw[thick,-Latex] (0,0,0) -- (0,0,1) node[anchor=south]{$e_3 = f_3$};

%draw a vector from origin to point (P) 
\draw[-stealth,color=red, -Latex] (O) -- node [below right] {$ r $} ($(\Px, \Py, \Pz) $);

%draw the angle \phi, and label it
%syntax: \tdplotdrawarc[coordinate frame, draw options]{center point}{r}{angle}{label options}{label}
\tdplotdrawarc{(O)}{0.2}{0}{\phivec}{anchor=north}{$\Omega_A t$}


%set the rotated coordinate system so the x'-y' plane lies within the
%"theta plane" of the main coordinate system
%syntax: \tdplotsetthetaplanecoords{\phi}
\tdplotsetthetaplanecoords{\phivec}

% draw the asteroid frame
% \draw[thick,->,tdplot_rotated_coords] (0,0,0) -- (1,0,0) node[anchor=north east]{$f_3$};
\draw[thick,-Latex,tdplot_rotated_coords] (0,0,0) -- (0,1,0) node[anchor=north west]{$f_1$};
% \draw[thick,->,tdplot_rotated_coords] (0,0,0) -- (0,0,1) node[anchor=north west]{$f_2$};

%draw some dashed arcs, demonstrating direct arc drawing
\draw[dashed,tdplot_rotated_coords] (\rvec,0,0) arc (0:90:\rvec);
\draw[dashed] (\rvec,0,0) arc (0:90:\rvec);

%set the rotated coordinate definition within display using a translation
%coordinate and Euler angles in the "z(\alpha)y(\beta)z(\gamma)" euler rotation convention
%syntax: \tdplotsetrotatedcoords{\alpha}{\beta}{\gamma}
\tdplotsetrotatedcoords{\phivec}{\thetavec}{0}

%translate the rotated coordinate system
%syntax: \tdplotsetrotatedcoordsorigin{point}
\tdplotsetrotatedcoordsorigin{(Ptikz)}

%use the tdplot_rotated_coords style to work in the rotated, translated coordinate frame
\draw[thick,tdplot_rotated_coords,-Latex] (0,0,0) -- (\b,0,0) node[anchor=north west]{$b_1$};
\draw[thick,tdplot_rotated_coords,-Latex] (0,0,0) -- (0,\b,0) node[anchor=west]{$b_2$};
\draw[thick,tdplot_rotated_coords,-Latex] (0,0,0) -- (0,0,\b) node[anchor=south]{$b_3$};

% transform the first dumbbell
% rot to main
\tdplottransformrotmain{\d}{0}{0};
\tdplottransformmainscreen{\tdplotresx}{\tdplotresy}{\tdplotresz};
\pgfmathsetmacro{\DMonescreenx}{\tdplotresx};
\pgfmathsetmacro{\DMonescreeny}{\tdplotresy};

\tdplottransformrotmain{-\d}{0}{0};
\tdplottransformmainscreen{\tdplotresx}{\tdplotresy}{\tdplotresz};
\pgfmathsetmacro{\DMtwoscreenx}{\tdplotresx};
\pgfmathsetmacro{\DMtwoscreeny}{\tdplotresy};

% transform P to screen coordinates and then add
\tdplottransformmainscreen{\Px}{\Py}{\Pz};
\pgfmathsetmacro{\Pscreenx}{\tdplotresx};
\pgfmathsetmacro{\Pscreeny}{\tdplotresy};

\draw[ultra thick, color=blue, tdplot_rotated_coords] (-\d, 0, 0) -- (\d, 0, 0);

\shade[ball color=red, tdplot_screen_coords, opacity=0.3] (\Pscreenx, \Pscreeny) circle (0.03);
\shade[ball color=blue,tdplot_screen_coords] ($(\Pscreenx, \Pscreeny) + (\DMonescreenx, \DMonescreeny) + (0, 0.015)$) circle (0.03);
\shade[ball color=blue,tdplot_screen_coords] ($(\Pscreenx, \Pscreeny) + (\DMtwoscreenx, \DMtwoscreeny) + (0, -0.015)$ ) circle (0.03);

\end{tikzpicture}


            }
            \end{scaletikzpicturetowidth}
        \end{column}
    \end{columns}
\end{frame}

\begin{frame}{Spacecraft Model}
    \begin{itemize}
        \item Spacecraft modeled as a rigid dumbbell
        \item Two masses \( m_1, m_2 \in \R^1 \) connected by a rigid massless link \( l \in \R^1\)
        \item Effectively captures the mass distribution of a spacecraft
    \end{itemize}

    \resizebox{\textwidth}{!}{%
        %% helper macros

\newcommand\pgfmathsinandcos[3]{%
  \pgfmathsetmacro#1{sin(#3)}%
  \pgfmathsetmacro#2{cos(#3)}%
}
\newcommand\LongitudePlane[3][current plane]{%
  \pgfmathsinandcos\sinEl\cosEl{#2} % elevation
  \pgfmathsinandcos\sint\cost{#3} % azimuth
  \tikzset{#1/.style={cm={\cost,\sint*\sinEl,0,\cosEl,(0,0)}}}
}
\newcommand\LatitudePlane[3][current plane]{%
  \pgfmathsinandcos\sinEl\cosEl{#2} % elevation
  \pgfmathsinandcos\sint\cost{#3} % latitude
  \pgfmathsetmacro\yshift{\cosEl*\sint}
  \tikzset{#1/.style={cm={\cost,0,0,\cost*\sinEl,(0,\yshift)}}} %
}

\newcommand\DrawLongitudeCircle[4][1]{
\LongitudePlane{\angEl}{#2}
\tikzset{current plane/.prefix style={scale=#1}}
% angle of "visibility"
\pgfmathsetmacro\angVis{
atan(sin(#2)*cos(\angEl)/sin(\angEl))} %
\draw[shift={(#3, #4)}][current plane]
(\angVis:1) arc (\angVis:\angVis+180:1);
\draw[shift={(#3, #4)}][current plane,dashed]
(\angVis-180:1)arc(\angVis-180:\angVis:1);
}
\newcommand\DrawLatitudeCircle[4][1]{
\LatitudePlane{\angEl}{#2}
\tikzset{current plane/.prefix style={scale=#1}}
\pgfmathsetmacro\sinVis{
sin(#2)/cos(#2)*sin(\angEl)/cos(\angEl)}
% angle of "visibility"
\pgfmathsetmacro\angVis{
asin(min(1,max(\sinVis,-1)))}
\draw[shift={(#3, #4)}][current plane]
(\angVis:1) arc (\angVis:-\angVis-180:1);
\draw[shift={(#3, #4)}][current plane,dashed]
(180-\angVis:1)arc(180-\angVis:\angVis:1);
}

%% document-wide tikz options and styles

\tikzset{%
  >=latex, % option for nice arrows
  inner sep=0pt,%
  outer sep=2pt,%
  mark coordinate/.style={inner sep=0pt,outer sep=0pt,minimum size=3pt,
    fill=black,circle}%
}

\tikzsetnextfilename{dumbbell}
\begin{tikzpicture} % "THE GLOBE" showcase

\def\R{2} % sphere radius
\def\angEl{35} % elevation angle
\def\angAz{-105} % azimuth angle
\def\length{4} % distance from COM to M_1
\def\opacity{0.2}

% centers of the masses
\coordinate (m1) at (\length, 0);
\coordinate (m2) at (-\length, 0);

\draw [ thick, dashed, -Latex] (m2) -- ($ (m1) + (3, 0) $) node [below right] {$b_1$};
\draw [thick, dashed, -Latex] (0, 0) -- (0, \R) node [above right] {$b_2$};
% TODO Make a macro to draw a sphere at a sphefic point
\filldraw[ball color=blue, opacity=\opacity] (m1) circle (\R);
\foreach \t in {-45, 0, 45} { \DrawLatitudeCircle[\R]{\t}{\length}{0} }
\foreach \t in {-30, -60,...,-150} { \DrawLongitudeCircle[\R]{\t}{\length}{0} }

\filldraw[ball color=blue, opacity=\opacity] (m2) circle (\R);
\foreach \t in {-60,-30,...,60} { \DrawLatitudeCircle[\R]{\t}{-\length}{0} }
\foreach \t in {-5,-35,...,-175} { \DrawLongitudeCircle[\R]{\t}{-\length}{0} }

% draw from origin to center of m1
\draw[thick,-Latex] node [below right] {$\zeta$} (0, 0) --  (m1);
\draw[thick,-Latex] (m1) -- ++(30:\R) node [above right] {$\eta$};
\end{tikzpicture}


    }
\end{frame}
\subsection{EOMs}



\begin{frame}{Equations of Motion}
    Hamilton's principle used to derive equations of motion in four forms:
    \begin{enumerate}
        \item Inertial vs. Relative 
        \item Lagrangian vs. Hamiltonian
    \end{enumerate}
    True motion of the system is a stationary point of the action integral
    \begin{align*}
        \mathcal{G} &= \int_{t_0}^{t_f} T(\dot q) - V(q) dt\\
        \pause
        \delta \mathcal{G} &= \int_{t_0}^{t_f} \delta T - \delta V dt = 0
    \end{align*}
\end{frame}

\begin{frame}{Inertial EOMs in Lagrangian Form}
    The kinetic and potential energy of the dumbbell in the inertial frame
    \begin{align*}
        T\parenth{\dot{\ipos}, \iangvel} &= \frac{1}{2} m \norm{\dot{\vc{x}}}^2 + \frac{1}{2} \tr{\vh{\Omega} J_d \vh{\Omega}^T} , \\
        V( \ipos, R ) &=  - m_1 U \parenth{R_A^T \parenth{\vc{x} + R \vc{\rho}_1}} - m_2 U \parenth{R_A^T \parenth{\vc{x} + R \vc{\rho}_2}}. 
    \end{align*}
    \pause
    The stationary point of the action integral gives:
    \begin{align*}
        \dot{\ipos} &= \ivel, \\
        \parenth{m_1 + m_2} \dot{\ivel} &= - \sum_{i=0}^2 m_i \aatt \deriv{U}{\apos_i} + u_f, \\
        \dot{\iatt} &= \iatt \iangvel, \\
        J \dot{\iangvel} + \hat{\iangvel} J \iangvel &= \sum_{i=0}^2 m_i \hat{\spos}_i \iatt^T \aatt \deriv{U}{\apos_i} + u_m. 
    \end{align*}
\end{frame}

\begin{frame}{Relative Equations of Motion}

\begin{itemize}
    \item Gravitational potential is a function of relative position
    \item Can derive EOMs in the rotating small body fixed frame
    \item Kinematics defined relative to the small body
\end{itemize}
\pause
\begin{align*}
    \dot{x}_r &= \rvel - \hat{\Omega}_A \rpos, \\
    \parenth{m_1 + m_2} \dot{v}_r  &= -\parenth{m_1 + m_2} \hat{\Omega}_A \rvel + m_1 \deriv{U}{z_1} + m_2 \deriv{U}{z_2}, \\
    \dot{R}_r &= \hat{\Omega}_r \ratt  - \hat{\Omega}_A \ratt , \\
    \dot{\parenth{\Jr  \rangvel}} + \hat{\Omega}_A \parenth{\Jr  \rangvel} &= m_1 \parenth{\ratt  \rho_1}^\wedge \deriv{U}{z_1} + m_2 \parenth{\ratt  \rho_2}^\wedge \deriv{U}{z_2}.
\end{align*}
\end{frame}

\subsection{Gravity}

\begin{frame}{Gravitational Models}
    \begin{center}
        \tikzsetnextfilename{newton_gravity}
\begin{tikzpicture}
    \coordinate (m1) at (0, 0);
    \coordinate (m2) at (8, 0);

    \shade[ball color=blue, opacity=0.7] (m1) circle (1.5);
    \shade[ball color=red, opacity=0.7] (m2) circle (1);

    \node[ above] at (m1) {$m_1$};
    \node[above] at (m2) {$m_2$};

    \draw[-Latex] (m1) -- ++(3, 0) node[below ] {$r$}; 
    \draw[-Latex] (m2) -- ++(-2, 0) node[above] {$F_g$};
\end{tikzpicture}

    \end{center}
    \pause
    \begin{itemize}
        \item Newton's law of gravity only applies to spherical bodies
        \item Variety of methods to represent irregular bodies
            \begin{itemize}
                \item Spherical Harmonic - not valid when close to body
                \item Constant Density Ellipsoid - only valid for ellipsoids
                \item \Emph{Polyhedron Potential} - function of shape model 
            \end{itemize}
    \end{itemize}
\end{frame}

\begin{frame}{Polyhedron Gravitation Model}

\begin{columns}
\begin{column}{0.5\textwidth}
\begin{itemize}
    \item Polyhedron: solid with flat polygonal faces
    \item Assumes a constant density
    \item Gravity depends on the shape model
\end{itemize}
\end{column}
\begin{column}{0.5\textwidth}
    \resizebox{\columnwidth}{!}{%
    \tikzsetnextfilename{polyhedron_faces}

\begin{tikzpicture}[line join=bevel,z=-5.5]
	% \draw[help lines] (-10,-10) grid (10,10); %grid
    \coordinate (A1) at (0,0,-3);
    \coordinate (A2) at (-3,0,0);
    \coordinate (A3) at (0,0,3);
    \coordinate (A4) at (3,0,0);
    \coordinate (B1) at (0,3,0);
    \coordinate (C1) at (0,-3,0);
    
    \coordinate (Acenter) at (-1, 1, 1);
    \coordinate (Bcenter) at (1, 1, 1);

    \draw (A1) -- (A2) -- (B1) -- cycle;
    \draw (A4) -- (A1) -- (B1) -- cycle;
    \draw (A1) -- (A2) -- (C1) -- cycle;
    \draw (A4) -- (A1) -- (C1) -- cycle;
    \draw [fill opacity=0.7,fill=green!80!blue] (A2) -- (A3) -- (B1) -- cycle;
    \draw [fill opacity=0.7,fill=orange!80!black] (A3) -- (A4) -- (B1) -- cycle;
    \draw [fill opacity=0.7,fill=green!30!black] (A2) -- (A3) -- (C1) -- cycle;
    \draw [fill opacity=0.7,fill=purple!70!black] (A3) -- (A4) -- (C1) -- cycle;

    \shade[ball color=blue] (A3) circle (0.1) node [below left] {$v_1$};
    \shade[ball color=blue] (B1) circle (0.1) node [above] {$v_2$};
    \shade[ball color=blue] (A2) circle (0.1) node [left] {$v_3$};

    \draw[ultra thick, -Latex] (A3) --node[right] {$h_1$}  (B1);
    \draw[ultra thick, -Latex] (B1) --node[above left] {$h_2$}  (A2);
    \draw[ultra thick, -Latex] (A2) --node[above ] {$h_3$}  (A3);
    \draw[ultra thick, -Latex] (Acenter) -- ++(-2, 2, 2) node[left] {$n_A$};
    \draw[ultra thick, -Latex] (Bcenter) -- ++(2, 2, 2) node[right] {$n_B$};
    

\end{tikzpicture}

}
\end{column}
\end{columns}
\pause
\begin{align*}
    U(\vc{r}) =& \frac{1}{2} G \sigma \sum_{e \in \text{edges}} \vc{r}_e \cdot \vc{E}_e \cdot \vc{r}_e \cdot L_e - \frac{1}{2}G \sigma \sum_{f \in \text{faces}} \vc{r}_f \cdot \vc{F}_f \cdot \vc{r}_f \cdot \omega_f 
\end{align*}
\end{frame}
