% -------------------- SYMBOLS ---------------------------------
\newglossaryentry{M}{
    type=symbols,
    name={\ensuremath{M}},
    sort=M,
    description={a \gls{matrix}}
}

\newglossaryentry{F}{
    type=symbols,
    name={\ensuremath{F}},
    sort=F,
    description={External Force}
}

\newglossaryentry{rho}{
    type=symbols,
    name={\ensuremath{\vec \rho}},
    sort=R,
    description={Range to target}
}

\newglossaryentry{view_axis}{
    type=symbols,
    name={\ensuremath{\vec d}},
    sort=D,
    description={Sensor optical axis}
}

\newglossaryentry{view_distance}{
    type=symbols,
    name={\ensuremath{d}},
    sort=D,
    description={Distance from the sensor, along the view axis, to the far plane defining the sensor viewing frustum},
    see={frustum,view_axis}
}

\newglossaryentry{up_axis}{
    type=symbols,
    name={\ensuremath{\vec u}},
    sort=U,
    description={Sensor up axis}
}

\newglossaryentry{fov_h}{
    type=symbols,
    name={\ensuremath{\alpha}},
    sort=A,
    description={Horizontal sensor field of view}
}

\newglossaryentry{fov_v}{
    type=symbols,
    name={\ensuremath{\beta}},
    sort=B,
    description={Vertical sensor field of view}
}

\newglossaryentry{sym:Ra}{
    type=symbols,
    name={\ensuremath{R_A}},
    description={Attitude of the asteroid. This rotation matrix transforms vectors from the asteroid fixed frame to the inertial frame.},
    nonumberlist=true
}

\newglossaryentry{sym:ix}{
    type=symbols,
    name={\ensuremath{\vc{x}}},
    sort=X,
    description={Inertial position of the spacecraft center of mass expressed in the inertial frame.},
    nonumberlist=true
}

\newglossaryentry{sym:iv}{
    type=symbols,
    name={\ensuremath{\vc{v}}},
    sort=V,
    description={Inertial velocity vector of spacecraft with respect to the inertial frame and expressed in the inertial frame.},
    nonumberlist=true
}

\newglossaryentry{sym:iR}{
    type=symbols,
    name={\ensuremath{R}},
    sort=R,
    description={Inertial attitude of the spacecraft with respect to the inertial frame. This rotation matrix transforms vectors from the spacecraft fixed frame to the inertial frame.},
    nonumberlist=true
}

\newglossaryentry{sym:iW}{
    type=symbols,
    name={\ensuremath{\vc{\Omega}}},
    sort=W,
    description={Angular velocity of the spacecraft with respect to the inertial frame and expressed in the spacecraft fixed frame.},
    nonumberlist=true
}
\newglossaryentry{sym:rx} {
    type=symbols,
    name={\ensuremath{\vc{\mathsf{x}}}},
    sort=X,
    description={Relative position vector of spacecraft with respect to asteroid and expressed in the asteroid fixed frame.},
    nonumberlist=true
}

\newglossaryentry{sym:rv}{
    type=symbols,
    name={\ensuremath{\vc{\mathsf{v}}}},
    sort=V,
    description={Relative velocity vector of spacecraft with respect to asteroid and expressed in the asteroid fixed frame.},
    nonumberlist=true
}

\newglossaryentry{sym:rR}{
    type=symbols,
    name={\ensuremath{\mathsf{R}}},
    sort=R,
    description={Relative attitude of the spacecraft with respect to the asteroid. This rotation matrix converts vectors in the spacecraft fixed frame to the asteroid fixed frame.},
    nonumberlist=true
}

\newglossaryentry{sym:rW}{
    type=symbols,
    name={\ensuremath{\mathsf{\Omega}}},
    sort=W,
    description={Angular velocity of the spacecraft with respect to the inertial frame and expressed in the asteroid fixed frame.},
    nonumberlist=true
}

\newglossaryentry{sym:rxvar}{
    type=symbols,
    name={\ensuremath{\mathsf{\chi}}},
    sort=X,
    description={Reduced variation of the relative spacecraft position.},
    nonumberlist=true
}
