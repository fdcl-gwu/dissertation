% !TEX root = ../../dissertation.tex

\chapter{Derivation of Relative Equations of Motion}\label{proof:relative_dumbbell_eoms}

\subsection{Relative variations}\label{ssec:relative_variations}
In~\cref{proof:inertial_dumbbell_eoms} the variations of the inertial configuration variables were derived.
Following a similar process the variations of the reduced variables must be derived such that they also satisfy the variations of the inertial variables.
These variations must be defined precisely to ensure that the configuration space is preserved properly. 
These variations will be critical for the following step of applying the Euler-Lagrage equations and deriving the equations of motion.
The variations of the reduced variables are obtained from the variations of the original variables given in~\cref{eq:inertial_pos_variation,eq:inertial_vel_variation,eq:inertial_att_variation,eq:inertial_ang_vel_variation} and the definition of the reduced variables given in~\cref{eq:relative_position,eq:relative_velocity,eq:relative_attitude,eq:relative_angular_velocity}.

The variation of the relative position given in~\cref{eq:relative_position} is defined as
\begin{align}
    \delta \gls{sym:rpos} &= \left. \diff{}{\epsilon} \right|_{\epsilon=0} \gls{sym:Ra}[^T] \gls{sym:ipos}^\epsilon ,\nonumber \\
                        &= \gls{sym:Ra}[^T] \delta \gls{sym:ipos} .\label{eq:relative_position_variation}
\end{align}
The variation of the relative attitude given in~\cref{eq:relative_attitude} is expressed as
\begin{align}
    \delta \gls{sym:ratt} &= \left. \diff{}{\epsilon} \right|_{\epsilon=0}  \gls{sym:Ra}[^T] \gls{sym:iatt}^\epsilon = \gls{sym:Ra}[^T] \delta \gls{sym:iatt}, \nonumber\\
                          &= \gls{sym:Ra}[^T] \gls{sym:iatt} \gls{sym:iattvar} = \gls{sym:ratt} \gls{sym:iattvar}, \label{eq:relative_attitude_variation_1}
\end{align}
where we used the fact that the asteroid attitude does not vary, \( \delta \gls{sym:Ra} = 0\), and the variation of the spacecraft attitude is given in~\cref{eq:inertial_att_variation}.
The variation of the spacecraft attitude \( \gls{sym:iattvar} \in \so \) is related to the relative variation \( \gls{sym:rattvar} \in \so \) by
\begin{align}
    \gls{sym:rattvar} &= \gls{sym:ratt} \gls{sym:iattvar} \gls{sym:ratt}[^T],\nonumber\\
                      &= \gls{sym:Ra}[^T] \gls{sym:iatt} \gls{sym:iattvar} \gls{sym:iatt}[^T] \gls{sym:Ra}.\label{eq:relative_attitude_variation}
\end{align}
We can invert~\cref{eq:relative_attitude_variation} to find
\begin{align}\label{eq:relative_attitude_variation_inverse}
    \gls{sym:iattvar} &= \gls{sym:iatt}[^T] \gls{sym:Ra} \gls{sym:rattvar} \gls{sym:Ra}[^T] \gls{sym:iatt}.
\end{align}
Using this, the varation of the relative attitude is derived by substituting~\cref{eq:relative_attitude_variation_inverse} into~\cref{eq:relative_attitude_variation_1} as
\begin{align}\label{eq:relative_attitude_matrix_variation}
    \delta \gls{sym:ratt} = \gls{sym:Ra}[^T] \gls{sym:iatt} \gls{sym:iattvar} = \gls{sym:rattvar} \gls{sym:Ra}[^T] \gls{sym:iatt} = \gls{sym:rattvar} \gls{sym:ratt}
\end{align}

The variation of the relative angular velocity is defined as follows:
\begin{align}\label{eq:relative_angular_velocity_variation_1}
    \parenth{\delta \gls{sym:rangvel}}^\wedge &= \left. \diff{}{\epsilon} \right|_{\epsilon=0} \parenth{ \gls{sym:ratt}[^\epsilon] \gls{sym:iangvel}[^\epsilon] }^\wedge, \nonumber\\
                                              &= \left. \diff{}{\epsilon} \right|_{\epsilon=0} \gls{sym:ratt}[^\epsilon] \hat{\glsentrytext{sym:iangvel}}^\epsilon \parenth{\gls{sym:ratt}[^\epsilon]}^T, \nonumber\\
                                              &= \delta \gls{sym:ratt} \hat{\gls{sym:iangvel}} \gls{sym:ratt}[^T] + \gls{sym:ratt} \delta \hat{\gls{sym:iangvel}} \gls{sym:ratt}[^T] + \gls{sym:ratt} \hat{\gls{sym:iangvel}} \delta \gls{sym:ratt}[^T],
\end{align}
where we used~\cref{eq:hatRxR} to arrive at the final form.
The variation \( \delta \gls{sym:iangvel} \) is defined in~\cref{eq:inertial_ang_vel_variation} and subsitituded into~\cref{eq:relative_angular_velocity_variation_1} to give
\begin{align}\label{eq:relative_angular_velocity_variation_2}
    \parenth{\delta \gls{sym:rangvel}}^\wedge &= \delta \gls{sym:ratt} \hat{\gls{sym:iangvel}} \gls{sym:ratt}[^T] + \gls{sym:ratt} \delta \hat{\gls{sym:iangvel}} \gls{sym:ratt}[^T] + \gls{sym:ratt}\hat{\gls{sym:iangvel}}\delta \gls{sym:ratt}[^T], \nonumber \\
                                              &= \gls{sym:rattvar} \gls{sym:ratt} \hat{\gls{sym:iangvel}} \gls{sym:ratt}[^T] + \gls{sym:ratt} \parenth{ - \gls{sym:iattvar} \hat{\gls{sym:iangvel}} + \hat{\gls{sym:iangvel}}\gls{sym:iattvar} + \dot{\gls{sym:iattvar}} } \gls{sym:ratt}[^T] + \gls{sym:ratt} \hat{\gls{sym:iangvel}} \gls{sym:ratt}[^T] \gls{sym:rattvar}[^T], \nonumber \\
                                              &= \gls{sym:rattvar} \gls{sym:ratt}\hat{\gls{sym:iangvel}}\gls{sym:ratt}[^T] - \gls{sym:ratt} \gls{sym:iattvar} \hat{\gls{sym:iangvel}} \gls{sym:ratt}[^T] + \gls{sym:ratt} \hat{\gls{sym:iangvel}} \gls{sym:iattvar} \gls{sym:ratt}[^T] + \gls{sym:ratt} \dot{\gls{sym:iattvar}} \gls{sym:ratt}[^T] + \gls{sym:ratt} \hat{\gls{sym:iangvel}} \gls{sym:ratt}[^T] \gls{sym:iattvar}[^T].
\end{align}
Using the hat map identity given in~\cref{eq:hatRxR} in~\cref{eq:relative_angular_velocity_variation_2} gives
\begin{align}\label{eq:relative_angular_velocity_variation_3}
\parenth{\delta \gls{sym:rangvel}}^\wedge &= \gls{sym:rattvar} \parenth{ \gls{sym:ratt} \gls{sym:iangvel}}^\wedge - \gls{sym:ratt} \gls{sym:iattvar} \hat{\gls{sym:iangvel}} \gls{sym:ratt}[^T] + \gls{sym:ratt} \hat{\gls{sym:iangvel}} \gls{sym:iattvar} \gls{sym:ratt}[^T] + \gls{sym:ratt} \dot{\eta}_r \gls{sym:ratt}[^T] + \parenth{\gls{sym:ratt} \gls{sym:iangvel} }^\wedge \gls{sym:rattvar}[^T].
\end{align}
We can further simplify~\cref{eq:relative_angular_velocity_variation_3} by using the following identities which make use of~\cref{eq:relative_attitude_variation,eq:hatRxR}:
\begin{align}
    \gls{sym:ratt} \hat{\gls{sym:iangvel}} \gls{sym:iattvar} \gls{sym:ratt}[^T] &= \gls{sym:ratt} \hat{\gls{sym:iangvel}} \gls{sym:ratt}[^T] \gls{sym:ratt} \gls{sym:iattvar} \gls{sym:ratt}[^T] = \parenth{ \gls{sym:ratt} \gls{sym:iangvel}}^\wedge \parenth{ \gls{sym:ratt} \gls{sym:iattvar} \gls{sym:ratt}[^T]} = \parenth{\gls{sym:ratt}\gls{sym:iangvel}}^\wedge \gls{sym:rattvar}, \label{eq:RSeta_identity_1}\\
    -\gls{sym:ratt} \gls{sym:iattvar} \hat{\gls{sym:iangvel}} \gls{sym:ratt}[^T] &= -\gls{sym:ratt} \gls{sym:iattvar} \gls{sym:ratt}[^T] \gls{sym:ratt} \hat{\gls{sym:iangvel}} \gls{sym:ratt}[^T] = - \parenth{\gls{sym:ratt}\gls{sym:iattvar}\gls{sym:ratt}[^T]} \parenth{\gls{sym:ratt} \gls{sym:iangvel}}^\wedge = -\gls{sym:rattvar} \parenth{ \gls{sym:ratt} \gls{sym:iangvel}}^\wedge. \label{eq:RSeta_identity_2}
\end{align}
Substituting~\cref{eq:RSeta_identity_1,eq:RSeta_identity_2} into~\cref{eq:relative_angular_velocity_variation_3} gives
\begin{align}\label{eq:relative_angular_velocity_variation_4}
    \parenth{\delta \gls{sym:rangvel}}^\wedge &= \gls{sym:rattvar} \parenth{\gls{sym:ratt}\gls{sym:iangvel}}^\wedge - \gls{sym:rattvar} \parenth{\gls{sym:ratt}\gls{sym:iangvel}}^\wedge + \parenth{\gls{sym:ratt}\gls{sym:iangvel}}^\wedge + \gls{sym:ratt}\dot{\gls{sym:iattvar}}\gls{sym:ratt}[^T] + \parenth{\gls{sym:ratt}\gls{sym:iangvel}}^\wedge \gls{sym:rattvar}[^T] ,\nonumber \\
                                              &= \gls{sym:ratt} \dot{\gls{sym:iattvar}} \gls{sym:ratt}[^T] .
\end{align}
We now need to define \( \dot{\gls{sym:iattvar}}\) in terms of the reduced variables.
From~\cref{eq:relative_attitude} and the product rule we can define
\begin{align}\label{eq:relative_attitude_derivative_1}
    \dot{R}_r = \dot{R}_A^T R + R_A^T \dot{R}.
\end{align}
From the kinematic equation for the rotation matrix we have
\begin{subequations}\label{eq:attitude_kinematics_derivative}
\begin{align}
    \dot{R}_A &= R_A \hat{\Omega}_A, \\
    \dot{R} &= R \hat{\Omega} = R \parenth{\gls{sym:ratt}[^T]\gls{sym:rangvel}}^\wedge, 
\end{align}
\end{subequations}
where we used~\cref{eq:relative_angular_velocity}.
Applying~\cref{eq:attitude_kinematics_derivative,eq:hatRxR} into~\cref{eq:relative_attitude_derivative_1} gives
\begin{align}
    \dot{R}_r &= \hat{\Omega}_A^T R_A^T \gls{sym:iatt} + \gls{sym:Ra}[^T] \gls{sym:iatt} \parenth{\gls{sym:ratt}[^T] \gls{sym:rangvel}}^\wedge, \nonumber \\
              &= \hat{\Omega}_A^T R_A^T \gls{sym:iatt} + \gls{sym:Ra}[^T] \gls{sym:iatt} \gls{sym:ratt}[^T] \hat{\Omega}_r \gls{sym:ratt}, \nonumber \\
              &= - \hat{\Omega}_A \gls{sym:ratt} + \hat{\Omega}_r \gls{sym:ratt}. \label{eq:relative_attitude_derivative}
\end{align}
From~\cref{eq:relative_attitude_variation} and the product rule, the time derivative of \( \gls{sym:rattvar} \) is given by
\begin{align}\label{eq:relative_attitude_variation_derivative}
    \dot{\eta}_r = \dot{R}_r \gls{sym:iattvar} \gls{sym:ratt}[^T] + \gls{sym:ratt}\dot{\gls{sym:iattvar}}\gls{sym:ratt}[^T] + \gls{sym:ratt}\gls{sym:iattvar}\dot{R}_r^T 
\end{align}
Substituting~\cref{eq:relative_attitude_derivative} into~\cref{eq:relative_attitude_variation_derivative} gives
\begin{align}\label{eq:relative_attitude_variation_derivative_2}
    \dot{\eta}_r &= \braces{-\hat{\Omega}_A \gls{sym:ratt} + \hat{\Omega}_r \gls{sym:ratt}}\gls{sym:iattvar}\gls{sym:ratt}[^T] + \gls{sym:ratt}\dot{\gls{sym:iattvar}}\gls{sym:ratt}[^T] + \gls{sym:ratt}\gls{sym:iattvar}\braces{\gls{sym:ratt}[^T] \hat{\Omega}_A - \gls{sym:ratt}[^T] \hat{\Omega}_r}.
\end{align}
Inverting~\cref{eq:relative_attitude_variation_derivative_2} and solving for \( \gls{sym:ratt} \dot{\gls{sym:iattvar}} \gls{sym:ratt}[^T] \) gives
\begin{align}\label{eq:eta_dot}
    \gls{sym:ratt} \dot{\gls{sym:iattvar}} \gls{sym:ratt}[^T] &= \braces{\Omega_A - \hat{\Omega}_r} \gls{sym:rattvar} + \gls{sym:rattvar} \braces{\hat{\Omega}_r - \hat{\Omega}_A} + \dot{\eta}_r .
\end{align}
Finally, we can now substitute~\cref{eq:eta_dot} into~\cref{eq:relative_angular_velocity_variation_4} to arrive at
\begin{align}\label{eq:relative_angular_velocity_variation}
    \parenth{\delta \gls{sym:rangvel}}^\wedge &= \gls{sym:ratt} \dot{\gls{sym:iattvar}} \gls{sym:ratt}[^T] = \dot{\eta}_r - \hat{\Omega}_r \gls{sym:rattvar} + \hat{\Omega}_A \gls{sym:rattvar} +\gls{sym:rattvar} \hat{\Omega}_r - \gls{sym:rattvar} \hat{\Omega}_A, \nonumber \\
                                              &= \dot{\eta}_r + \gls{sym:rattvar}\braces{\hat{\Omega}_r - \hat{\Omega}_A} + \braces{\hat{\Omega}_A - \hat{\Omega}_r}\gls{sym:rattvar} . 
\end{align}

\subsubsection{Relative velocity variation}
The variation of the relative linear velocity is computed from~\cref{eq:relative_velocity} as
\begin{align}\label{eq:relative_velocity_variation_1}
    \delta \gls{sym:rvel} &= \left. \diff{}{\epsilon} \right|_{\epsilon=0} \gls{sym:Ra}[^T] \dot{x}^\epsilon , \nonumber \\
                          &= R_A^T \delta \dot{\gls{sym:ipos}},
\end{align}
since we assume that the asteroid is in a constant rate of rotation.
We can redefine~\cref{eq:relative_velocity_variation_1} in terms of the relative variables by finding the derivative of~\cref{eq:relative_position_variation} as
\begin{align}\label{eq:relative_position_variation_derivative_1}
    \delta \dot{x}_r &= \dot{R}_A^T \delta \gls{sym:ipos} + \gls{sym:Ra}[^T] \parenth{\delta \dot{\gls{sym:ipos}}}, \nonumber \\
                     &= \parenth{\gls{sym:Ra} \hat{\Omega}_A}^T \delta\gls{sym:ipos} + \gls{sym:Ra}[^T] \parenth{\delta \dot{\gls{sym:ipos}}}.
\end{align}
Rearranging~\cref{eq:relative_position_variation_derivative_1} for \(\delta \dot{\gls{sym:ipos}} \) and substituting into~\cref{eq:relative_velocity_variation_1} gives
\begin{align}\label{eq:relative_velocity_variation}
    \delta \gls{sym:rvel} = \delta \dot{x}_r + \hat{\Omega}_A \delta \gls{sym:rpos}.
\end{align}

\subsubsection{Summary: Relative Variations}
In summary, the variation of the relative variables are defined as
\begin{align}
    \delta \gls{sym:rpos} &= \gls{sym:Ra}[^T] \delta \gls{sym:ipos}, \\
    \delta \gls{sym:ratt} &= \gls{sym:rattvar} \gls{sym:ratt} ,\\
    \parenth{\delta \gls{sym:rangvel}}^\wedge&= \dot{\eta}_r - \hat{\Omega}_r \gls{sym:rattvar} + \hat{\Omega}_A \gls{sym:rattvar} + \gls{sym:rattvar} \hat{\Omega}_R - \gls{sym:rattvar} \hat{\Omega}_A, \\
    \delta \gls{sym:rvel} &= \dot{x}_r + \hat{\Omega}_A \delta \gls{sym:rpos} .
\end{align}
