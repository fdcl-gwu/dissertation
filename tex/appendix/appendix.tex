% !TEX root = ../dissertation.tex

\chapter{Another Kinetic Energy derivation}
Another way to derive this relationship is to define the kinetic energy of the system as
\begin{align}\label{eq:T_prod}
    T = \frac{1}{2} \sum_{i=1}^2 m_i \dot{\vecbf{r}}_i \cdot \dot{\vecbf{r}}_i ,
\end{align}
where \( \dot{\vecbf{r}}_i \) is the inertial velocity of each mass of the dumbbell.
The inertial velocity is also defined as
\begin{align}\label{eq:inertial_vel}
    \dot{\vecbf{r}}_i = \dot{\vecbf{x}} + \dot{\vecbf{\rho}}_i 
\end{align}
Substituting~\cref{eq:inertial_vel} into~\cref{eq:T_prod} and expanding results in
\begin{align*}
    T = \frac{1}{2} \sum_{i=1}^2 m_i \dot{\vecbf{x}} \cdot \dot{\vecbf{x}}  + \sum_{i=1}^2 m_i \dot{\vecbf{x}} \cdot \dot{\vecbf{\rho}}_i + \frac{1}{2} \sum_{i=1}^2 m_i \dot{\vecbf{\rho}}_i \cdot \dot{\vecbf{\rho}}_i .
\end{align*}
As previously, the second term vanishes by the definition of the center of mass
\begin{align*}
    \sum m_i \dot{\vecbf{\rho}}_i &= 0, \\
    \sum m_i \vecbf{\rho} &= 0 .
\end{align*}
We define the total mass as \( m = \sum_{i=1}^2 m_i \) and rewrite~\cref{eq:T_prod} as 
\begin{align*}
    T = \frac{1}{2} m \dot{\vecbf{x}} \cdot \dot{\vecbf{x}} + \frac{1}{2} \sum_{i=1}^2 m_i \dot{\vecbf{\rho}}_i \cdot \dot{\vecbf{\rho}}_i .
\end{align*}
We assume that the vector \( \vecbf{\rho}_i \) is defined in the body fixed, \(\vecbf{b}_i\), of the dumbbell.
The inertial velocity \( \dot{\vecbf{\rho}}_i \) is defined as
\begin{align*}
    \dot{\vecbf{\rho}}_i = \vecbf{\Omega} \times \vecbf{\rho}_i ,
\end{align*}

where \( \vecbf{\omega} \) is the angular velocity of the dumbbell with respect to the inertial frame and defined in the body frame, \( \vecbf{b}_i\).
The quadratic term is then given by
\begin{align}
    \dot{\vecbf{\rho}}_i \cdot \dot{\vecbf{\rho}}_i &= \dot{\vecbf{\rho}}_i \cdot \parenth{\vecbf{\omega} \times \vecbf{\rho}} ,\nonumber \\
    &= \vecbf{\omega} \cdot \parenth{\vecbf{\rho}_i \times \dot{\vecbf{\rho}}_i } ,
\end{align}
where we used the cyclic permutation identity, \( \vecbf{a} \cdot \parenth{\vecbf{b} \times \vecbf{c}} = \vecbf{b} \cdot \parenth{\vecbf{c} \times \vecbf{a}}  = \vecbf{c} \cdot \parenth{\vecbf{a} \times \vecbf{b}}\).
The kinetic energy then becomes
\begin{align}
    T &= \frac{1}{2} \sum_{i=1}^2 m_i \dot{\vecbf{x}} \cdot \dot{\vecbf{x}}  + \sum_{i=1}^2 m_i \dot{\vecbf{x}} \cdot \dot{\vecbf{\rho}}_i + \frac{1}{2} \vecbf{\omega} \cdot \parenth{\sum_{i=1}^2 \vecbf{\rho}_i \times m_i \dot{\vecbf{\rho}}_i } , \nonumber\\
    T &= \frac{1}{2} m \dot{\vecbf{x}} \cdot \dot{\vecbf{x}} + \frac{1}{2} \vecbf{\omega}\cdot \vecbf{H} , \nonumber \\
    T &= \frac{1}{2} m \dot{\vecbf{x}} \cdot \dot{\vecbf{x}} + \frac{1}{2} \vecbf{\omega}\cdot J \vecbf{\omega} .
\end{align}


