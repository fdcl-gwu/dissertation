% !TEX root = ../../dissertation.tex

\chapter{Properties of the trace}\label{chap:trace}

The trace of \( A \in \R^{n \times n} \) is defined as
\begin{align}
    \tr{A} = \sum_i^n \bracket{A}_{ii},
\end{align}
where \( \bracket{A}_{ij} \) denotes the element of \( A \) in row \( i \) and column \( i \).
The trace can be considered as the scalar sum of the main diagonal of a square matrix.
It can be shown that the following relationships hold~\cite{petersen2008,strang2008}:
\begin{align}
    \tr{A} &= \sum_i^n \lambda_i, \label{eq:treig}\\
    \tr{A} &= \tr{A^T}, \label{eq:trAT} \\
    \tr{A B} &= \tr{B A} = \tr{B^T A^T} = \tr{A^T B^T}, \label{eq:trAB} \\
    \tr{A^T B } &= \sum_{p, q=1}^3 \bracket{A}_{pq} \bracket{B}_{pq}, \label{eq:trATB} \\
    \tr{A+B} &= \tr{A} + \tr{B}, \label{eq:trApB} \\
    \tr{ABC} &= \tr{BCA} = \tr{CAB}, \label{eq:trABC}\\
    \tr{\vc{x}\vc{x}^T} &= \vc{x}^T \vc{x}, \label{eq:trxxT} \\
    \tr{PQ} &= 0, \label{eq:trPQ}
\end{align}
for square matrices \( A, B \in \R^{n \times n} \), vector \( x \in \R^n\), skew-symmetric matrix \( P = -P^T \in \R^{n\times n} \), symmetric matrix \( Q = Q^T \in \R^{n \times n} \), and eigenvalues of \( A \) \( \lambda_i \in \R \).
