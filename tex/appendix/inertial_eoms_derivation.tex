% !TEX root = ../../dissertation.tex

\chapter{Derivation of Inertial Equations of Motion}\label{proof:inertial_dumbbell_eoms}
The Lagrangian function \( L \in \R \) is a scalar function of the generalized coordinates and their velocities..
The Lagrangian is defined as the total system kinetic energy minus the total system potential energy:
\begin{align*}
    L(q, \dot{q} ) = T(q, \dot{q}) - U (q) ,
\end{align*}
where \( T(q, \dot{q} \) is the kinetic energy of the system while \( U(q) \) is the potential energy of the system.

The derivation is simplified to the case of two spherical masses, but can be extended to any arbitrary configuration of masses.
The kinetic and potential energy of the dumbbell model is dupblicated from~\cref{eq:dumbbell_kinetic_and_potential_energy} 
\begin{align*}
    T &= \frac{1}{2} m \norm{\dot{\vb{x}}}^2 + \frac{1}{2} \tr{\vh{\Omega} J_d \vh{\Omega}^T} , \\
    V( \vecbf{x}, R ) &=  - m_1 U \parenth{R_A^T \parenth{\vecbf{x} + R \vecbf{\rho}_1}} - m_2 U \parenth{R_A^T \parenth{\vecbf{x} + R \vecbf{\rho}_2}} .
\end{align*}
The configuration space of the motion is \( \SE \), and as a result the variations must be correctly defined to respect the geometry of the configuration space.
Consider the variations of \( \vc{x}, \dot{\vc{x}} \) as
\begin{align}
    \vc{x}^\epsilon &= \vc{x} + \epsilon \delta \vc{x} + \mathcal{O}(\epsilon^2), \label{eq:inertial_pos_variation} \\
    \dot{\vc{x}}^\epsilon &= \dot{\vc{x}} \delta \dot{\vc{x}} + \mathcal{O}(\epsilon^2), \label{eq:inertial_vel_variation}
\end{align}
where \( \delta \vc{x}, \delta \dot{\vc{x}} \in \R^3 \) are the infinitesimal variations of the position and velocity which vanish at the endpoints.
In other words,
\begin{align*}
    \delta \vc{x} (t_0) &= \delta \vc{x} (t_f ) = 0, \\
    \delta \dot{\vc{x}} (t_0) &= \delta \dot{\vc{x}} (t_f) = 0 .
\end{align*}

