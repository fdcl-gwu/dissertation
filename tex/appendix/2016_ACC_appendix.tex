\chapter{2016 ACC Appendicies}

\subsection{Proof of~\Cref{prop:config_error}}\label{proof:config_error}
To prove~\cref{item:prop_psi_psd} we note that~\cref{eqn:A} is a positive definite function about \( R = R_d \)~\cite{bullo2004}.
The constraint angle is assumed \( \ang{0} \leq \theta \leq \ang{90} \) such that \( 0 \leq \cos \theta \).
The term \( r^T R^T v \) represents the cosine of the angle between the body fixed vector \( r \) and the inertial vector \( v \). 
It follows that
\begin{align*}
	0 \leq  \frac{\cos \theta -  r^T R^T v}{1 + \cos \theta} \leq 1 ,
\end{align*}
for all \( R \in \SO \). 
As a result, its negative logarithm is always positive and from~\cref{eqn:B}, \(1 < B\).
The error function \( \Psi = A B \) is composed of two positive terms and is therefore also positive definite, and it is minimized at \( R = R_d \).

Next, we consider~\cref{item:prop_era}.
The variation of~\cref{eqn:A} is taken with respect to \( \delta R = R \hat \eta \) as
\begin{align*}
	\dirDiff{A}{R} &= \eta \cdot \frac{1}{2} \left( G R_d^T R - R^T R_d G\right)^\vee ,
%	&= -\frac{1}{2} \tr{G R_d^T \left( R \hat{\eta}\right)} \\
\end{align*}
where we used~\cref{eqn:hat1}.

A straightforward application of the chain and product rules of differentiation allows us to show~\cref{item:prop_erb} as
\begin{align*}
	\dirDiff{B}{R} &=  \eta \cdot \frac{ - \left( R^T v\right)^\vee r}{\alpha \left(\cos \theta - r^T R^T v \right)} ,
%	-\frac{1}{\alpha} \left[\left(\frac{1+\cos \theta}{\cos \theta - r^T R^T v} \right) \left(\eta \cdot \frac{\left( R^T v\right)^\vee r}{1+\cos \theta} \right) \right]
\end{align*}
where the scalar triple product~\cref{eqn:STP} was used.

%To show~\cref{item:prop_crit}, we determine the critical points of~\cref{eqn:eRA}.
%The attitude error between \( R \) and \( R_d \) is defined by \( Q \in \SO \) as
%\begin{align*}
%	Q = R_d^T R = 
%	\begin{bmatrix}
%		q_{11} & q_{12} & q_{13} \\
%		q_{21} & q_{22} & q_{23} \\
%		q_{31} & q_{32} & q_{33}
%	\end{bmatrix} .
%\end{align*}
%Equating \( e_{R_{A}} = 0 \) yields the following equivalent relations
%\begin{subequations}
%\begin{align}
%	g_3 q_{32} &= g_2 q_{23} , \\
%	g_1 q_{13} &= g_3 q_{31} , \\
%	g_2 q_{21} &= g_1 q_{12} .
%\end{align}
%\end{subequations}
%These relations allow one to redefine the attitude error \( Q \) in terms of the independent upper diagonal elements.
%Noting that \( Q \in \SO \) is an orthonormal matrix, with unit length row and columns gives
%\begin{subequations}
%\begin{align}
%	\parenth{1 - \frac{g_1^2}{g_2^2}} q_{12}^2 + \parenth{1 - \frac{g_1^2}{g_3^2}}q_{13}^2 &= 0 , \\
%	\parenth{1 - \frac{g_1^2}{g_2^2}} q_{12}^2 - \parenth{1 - \frac{g_2^2}{g_3^2}} q_{23}^2 &= 0 , \\
%	\parenth{1-\frac{g_1^2}{g_3^2}} q_{13}^2 + \parenth{1 - \frac{g_2^2}{g_3^2}} q_{23}^2 &= 0 .
%\end{align}
%\end{subequations}
%As a result, all the off diagonal terms of \( Q = R_d^T R \) are zero, namely \( q_{12}= q_{13} = q_{23} = 0 \).
%Since \( Q \in \SO \), the diagonal elements \( q_{11}, q_{22}, \text{ and } q_{33} \) are either \( +1 \) or \(-1\).
The critical points of \( e_{R_A} \) are derived in~\cite{bullo2004}.
There are four critical points of \( e_{R_A} \), the desired attitude \( R_d \) as well as rotations about each body fixed axis by \ang{180}.
The repulsive error vector \( e_{R_{B}} \) is zero only when the numerator \( \parenth{R^T v}^\wedge r = 0 \). 
This condition only occurs if the desired attitude results in the body fixed vector \( r \) becoming aligned with \(R^T v \) while simultaneously satisfying \(\braces{R_d} \cup \braces{R_d \exp(\pi \hat{s}} \) for \( s \in \braces{e_1, e_2, e_3} \).
Since we assume the system will not operate in violation of the constraints, the addition of the barrier function does not add additional critical points to the control system.
The desired equilibrium is \( e_R = 0 \) and \( A = 0\).
The proof of \( \norm{e_{R_A}} \) given by~\cref{item:prop_era_upbound} is available in~\cite{LeeITCST13}.
\subsection{Proof of~\Cref{prop:error_dyn}}\label{proof:error_dyn}
From the kinematics~\cref{eqn:Rdot} and noting that \( \dot{R}_d = 0 \) the time derivative of \( R_d^T R \) is given as
\begin{gather*}
	\diff{}{t} \parenth{R_d^T R} = R_d^T R \hat{e}_\Omega .
\end{gather*}
Applying this to the time derivative of~\cref{eqn:A} gives
\begin{gather*}
	\diff{}{t} (A) = -\frac{1}{2} \tr{G R_d^T R \hat{e}_\Omega} .
\end{gather*}
Applying~\cref{eqn:hat1} into this shows~\cref{eqn:A_dot}.
Next, the time derivative of the repulsive error function is given by
\begin{gather*}
	\diff{}{t} (B) = \frac{r^T \parenth{\hat{\Omega} R^T} v}{\alpha \parenth{r^T R^T v - \cos \theta}} .
\end{gather*}
Using the scalar triple product, given by~\cref{eqn:STP}, one can reduce this to~\cref{eqn:B_dot}.
The time derivative of the attractive attitude error vector, \( e_{R_A} \), is given by
\begin{gather*}
	\diff{}{t} ( e_{R_A}) = \frac{1}{2} \parenth{\hat{e}_\Omega R^T R_d G + (R^T R_d G)^T \hat{e}_\Omega}^\vee .
\end{gather*}
Using the hat map property given in~\cref{eqn:xAAx} this is further reduced to~\cref{eqn:eRA_dot,eqn:E}.

We take the time derivative of the repulsive attitude error vector, \( e_{R_B} \), as
\begin{gather*}
	\diff{}{t}( e_{R_B} )= a \Omega v^T R r - a R^T v \Omega^T r + b R^T \hat{v} R r ,
\end{gather*}
with \( a \in \R \) and \( b \in \R\) given by 
\begin{gather*}
	a = \bracket{\alpha \parenth{r^T R^T v - \cos \theta}}^{-1} , \,
	b = \frac{r^T \hat{\Omega} R^T v}{\alpha \parenth{r^T R^T v - \cos \theta}^2} .
\end{gather*}
Using the scalar triple product from~\cref{eqn:STP} as \( r \cdot \Omega \times \parenth{R^T v} = \parenth{R^T v} \cdot r \times \Omega \) gives~\cref{eqn:eRB_dot,eqn:F}.

We show the time derivative of the configuration error function as
\begin{gather*}
	\diff{}{t} (\Psi) = \dot{A} B + A \dot{B} .
\end{gather*}
A straightforward substitution of~\cref{eqn:A_dot,eqn:B_dot,eqn:A,eqn:B} into this and appplying~\cref{eqn:eR} shows~\cref{eqn:psi_dot}.
We show~\cref{eqn:eW_dot} by rearranging~\cref{eqn:Wdot} as 
\begin{align*}
	\diff{}{t} e_\Omega = \dot{\Omega} = J^{-1} \parenth{u - \Omega \times J \Omega + W(R,\Omega) \Delta } .
\end{align*}

\subsection{Proof of~\Cref{prop:att_control}}\label{proof:att_control}
Consider the following Lyapunov function:
\begin{gather}
	\mathcal{V} = \frac{1}{2} e_\Omega \cdot J e_\Omega + k_R \Psi(R,R_d) . \label{eqn:v_nodist}
\end{gather}
From~\cref{item:prop_psi_psd} of~\Cref{prop:config_error}, \(\mathcal{V} \geq 0 \).
Using~\cref{eqn:eW_dot,eqn:psi_dot} with \( \Delta = 0 \), the time derivative of \( \mathcal{V} \) is given by
\begin{align}
	\dot{\mathcal{V}} &= -k_\Omega \norm{e_\Omega}^2 . \label{eqn:vdot_nodist}
%	e_\Omega^T \parenth{ u - \Omega \times J \Omega } + k_R e_R \cdot e_\Omega \nonumber \\
\end{align}
Since \( \mathcal{V} \) is positive definite and \( \dot{\mathcal{V}} \) is negative semi-definite, the zero equilibrium point \( e_R, e_\Omega \) is stable in the sense of Lyapunov. 
This also implies \( \lim_{t\to\infty} \norm{e_\Omega} = 0 \) and \( \norm{e_R} \) is uniformly bounded, as the Lyapunov function is non-increasing. From \refeqn{eRA_dot} and \refeqn{eRB_dot}, $\lim_{t\to\infty} \dot e_R =0$. 
One can show that \( \norm{\ddot{e}_R} \) is bounded.
From Barbalat's Lemma, it follows \( \lim_{t\to\infty}\norm{\dot{e}_R} = 0 \)~\cite[Lemma 8.2]{khalil1996}. 
Therefore, the equilibrium is asymptotically stable. 
	
Furthermore, since \( \dot{\mathcal{V}} \leq 0 \) the Lyapunov function is uniformly bounded which implies 
\begin{align*}
	\Psi(R(t)) \leq \mathcal{V}(t) \leq \mathcal{V}(0) .
\end{align*}
In addition, the logarithmic term in~\cref{eqn:B} ensures \( \Psi(R) \to \infty \) as \( r^T R^T v \to \cos \theta \).
Therefore, the inequality constraint is always satisfied given that the desired equilibrium lies in the feasible set.
	
\subsection{Proof of~\Cref{prop:eR_dot_bound}}\label{proof:eR_dot_bound}
%	We first show~\cref{eqn:A_bound,eqn:B_bound} through a proof by contradiction by noting that given the selected domain \( D \) we know the configuration error function is bounded by \( \Psi = A(R) B(R) < \psi\).
%	If we assume that \( A > \psi \) this means that \( B < 1 \) in order to satisfy the domain \( \Psi < \psi \).
%	However, this would be in contradiction with~\Cref{proof:config_error} as \( B > 1 \), therefore \( A \) must always be less than \( \psi \).
%	Next, we show that since the domain \( D \) does not contain the constraint region there is an upper bound of \(B\).
%	We know that \( B \to \infty \) as the constraint is neared. 
%	We assume that \( B > \psi \) which implies that \( A < 1 \).
%	As \( \psi \) is increased this implies that the system is approaching the constraint and that \( A \to 0 \) in order to remain in the domain \( D \).
%	However, since the chosen domain is assumed to exclude the constraint this situation is not possible and \( B < \psi \).
	The selected domain ensures that the configuration error function is bounded \( \Psi < \psi \).
	This implies that that both \( A(R) \) and \( B(R) \) are bounded by constants \( c_A c_B < \psi < h_1\).
	Furthermore, since \( \norm{B} > 1 \) this ensures that \( c_A, c_B < \psi\) and shows~\cref{eqn:AB_bound}.
	
	Next, we show~~\cref{eqn:E_bound,eqn:F_bound} using the Frobenius norm.
	The Frobenius norm \( \norm{E}_F \) is given in~\cite{LeeITCST13} as
	\begin{gather*}
		\norm{E}_F = \sqrt{\tr{E^T E}} = \frac{1}{2} \sqrt{\tr{G^2} + \tr{R^T R_d G}^2} .
	\end{gather*}
	Applying Rodrigues' formula and the Matlab symbolic toolbox, this is simplified to
	\begin{gather*}
		\norm{E}^2_F \leq \frac{1}{4} \parenth{\tr{G^2} + \tr{G}^2} \leq \frac{1}{2} \tr{G}^2 ,
	\end{gather*}
	which shows~\cref{eqn:E_bound}, since \( \norm{E} \leq \norm{E}_F \).
	
	To show~\cref{eqn:F_bound}, we apply the Frobenius norm \( \norm{F}_F \):
	\begin{gather*}
		\norm{F}_F = \frac{1}{\alpha ^2 \parenth{r^T R^T v - \cos \theta}^2} \left[\tr{a^T a} - 2 \tr{a^T b} \right. \\
		\left.+ 2 \tr{a^T c } + \tr{b^T b}  - 2 \tr{b^T c} + \tr{c^T c}\right] .
	\end{gather*}
	where the terms \( a, b, \text{ and } c \) are given by
	\begin{gather*}
		a = r^T R r I , \quad	b = R^T v r^T , \quad c = \frac{R^T \hat{v} R r v^T R \hat{r}}{r^T R^T v - \cos \theta}.
	\end{gather*}
	A straightforward computation of \( a^T a \) shows that
	\begin{gather*}
		\tr{a^T a} = \parenth{v^T R r}^2 \tr{I} \leq 3 \beta^2 ,
	\end{gather*}
	where we used the fact that \( v^T R r = r^T R^T v < \beta \) from our given domain.
	Similarly, one can show that \( \tr{a^T b} \) is equivalent to
	\begin{gather*}
		\tr{a^T b} = v^T R r \tr{R^T v r^T} = \parenth{v^T R r}^2 \leq \beta^2 ,
	\end{gather*} 
	where we used the fact that \( \tr{x y^T} = x^T y \).
	The product \( \tr{a^T c} \) is given by
	\begin{gather*}
		\tr{a^T c} = \frac{v^T R r}{r^T R^T v - \cos \theta} \tr{\parenth{R^T v}^\vee \parenth{r v^T R} \hat{r} } ,
	\end{gather*}
	where we used the hat map property~\cref{eqn:RxR}.
	One can show that \(\mathrm{tr}[a^T c] \leq 0 \) over the range \( -1 \leq v^T R r \leq \cos \theta \). 
	Next, \( \tr{b^T b}\) is equivalent to
	\begin{gather*}
		\tr{b^T b} = \tr{r v^T R R^T v r^T} = 1 ,
	\end{gather*}
	since \( r,v \in \Sph^2\).
	Finally, \( \tr{c^T c} \) is reduced to
	\begin{gather*}
		\tr{c^T c} = \tr{\hat{r} R^T v r^T \bracket{-I + R^T v v^T R} r v^T R \hat{r}} ,
	\end{gather*}
	where we used the fact that \( \hat{x}^2 = - \norm{x}^2 I + x x^T\).
	Expanding and collecting like terms gives
	\begin{gather*}
		\tr{c^T c } = \frac{1 - 2\parenth{v^T R r}^2 + \parenth{v^T R r}^4}{\parenth{r^T R^T v - \cos \theta}^2} . 
	\end{gather*}
	Using the the given domain \( r^T R^T v \leq \beta \) gives the upper bound~\cref{eqn:F_bound}.
	The bound on \( e_{R_A} \) is given in~\cref{eqn:psi_lower_bound} while \( e_{R_B} \) arises from the definition of the cross product \( \norm{a \times b} = \norm{a} \norm{b} \sin \theta \).
	Finally, we can find the upper bound~\cref{eqn:eR_dot} as
	\begin{gather*}
		\norm{\dot{e}_R} \leq \parenth{\norm{B} \norm{E} + 2 \norm{e_{R_A}} \norm{e_{R_B}} + \norm{A}\norm{F}} \norm{e_\Omega} \, .
	\end{gather*}
	Using~\crefrange{eqn:AB_bound}{eqn:eRB_bound} one can define \( H \) in terms of known values.
\subsection{Proof of~\Cref{prop:adaptive_control}}\label{proof:adaptive_control}
	Consider the Lyapunov function \( \mathcal{V} \) given by
	\begin{gather}
		\mathcal{V} = \frac{1}{2} e_\Omega \cdot J e_\Omega + k_R \Psi + c J e_\Omega \cdot e_R + \frac{1}{2 k_\Delta} e_\Delta \cdot e_\Delta , \label{eqn:v_adapt}
	\end{gather}
	over the domain \( D \) in~\cref{eqn:domain}.
	From~\Cref{prop:eR_dot_bound}, the Lyapunov function is bounded in \( D \) by
	\begin{gather}
		\mathcal{V} \leq z^T W z , \label{eqn:v_upper_bound}
	\end{gather}
	where \( e_\Delta = \Delta - \bar{\Delta} \), \( z = [\|e_R\|,\|e_\Omega\|,\|e_\Delta\|]^T\in\R^3 \) and the matrix \(W \in \R^{3 \times 3}\) is given by
	\begin{gather*}
		W = \begin{bmatrix}
			k_R \psi & \frac{1}{2} c \lambda_M & 0 \\
			\frac{1}{2} c \lambda_M & \frac{1}{2} \lambda_M & 0 \\
			0 & 0 & \frac{1}{2 k_\Delta}
		\end{bmatrix} .
	\end{gather*}
	The time derivative of \( \mathcal{V}\) with the control inputs~\cref{eqn:adaptive_control} is given by
	\begin{align}
		\dot{\mathcal{V}} =& - k_\Omega e_\Omega^T e_\Omega + \parenth{e_\Omega + c e_R}^T W e_\Delta - k_R c e_R^T e_R \nonumber\\
		&- k_\Omega c e_R^T e_\Omega + c J e_\Omega^T \dot{e}_R - \frac{1}{k_\Delta} e_\Delta^T \dot{\bar{\Delta}} , \label{eqn:vdot}
	\end{align}
	where we used \( \dot{e}_\Delta = - \dot{\bar{\Delta}} \).
	The terms linearly dependent on \( e_\Delta\) are combined with~\cref{eqn:delta_dot} to yield
	\begin{align*}
		 e_\Delta^T \parenth{W^T \parenth{e_\Omega + c e_R} - \frac{1}{k_\Delta} \dot{\bar{\Delta}}} = 0 . 
	\end{align*}
	Using~\Cref{prop:eR_dot_bound} an upper bound on \( \dot{\mathcal{V}} \) is written as
	\begin{gather*}
		\dot{\mathcal{V}} \leq -\zeta^T M \zeta ,
	\end{gather*}
	where $\zeta=[\|e_R\|,\|e_\Omega\|]\in\R^2$, and the matrix \( M \in \R^{2 \times 2} \) is 
	\begin{gather}
		M = \begin{bmatrix}
			k_R c & \frac{k_\Omega c}{2} \\
			\frac{k_\Omega c}{2} & k_\Omega - c \lambda_M H
		\end{bmatrix} .
	\end{gather}
	If \( c \) is chosen such that~\cref{eqn:c_bound} is satisfied the matrix \( M \) is positive definite.
	This implies that $\dot{\mathcal{V}}$ is negative semidefinite and $\lim_{t\to\infty} \zeta=0$. 
	As the Lyapunov function is non-increasing $z$ is uniformly bounded. 
