\chapter{Nonlinear Geometric Controller Proofs}
This appendix presents the detailed proofs of the propositions given in~\cref{sec:se3_control}..

\section{Proof of~\Cref{prop:attitude_control_configuration_error}}\label{proof:attitude_config_error}

\subsection{Proof of~\Cref{item:prop_A_psd} and~\Cref{item:prop_eRA}}\label{proof:prop_psi_psd}
To prove~\cref{item:prop_psi_psd}, we first seek to show that \( A(R_d) \) is a critical point by showing that the first derivative is zero at \( R_d \).
The infinitesimal variation of a rotation matrix is defined as
\begin{align*}
    \delta R = \left. \diff{}{\epsilon} \right|_{\epsilon=0} R \exp{\epsilon \hat{\eta}} = R \hat{\eta}.
\end{align*}
Using this, the variation of~\cref{eq:A} is taken with respect to \( R \) as
\begin{align*}
	\dirDiff{A}{R} &= \eta \cdot \frac{1}{2} \left( G R_d^T R - R^T R_d G\right)^\vee ,
%	&= -\frac{1}{2} \tr{G R_d^T \left( R \hat{\eta}\right)} \\
\end{align*}
where we used~\cref{eqn:hat1} to achieve the simplified form.
\Cref{eq:dirDiff_A} also proves~\cref{item:prop_eRA}.
Evaluating~\cref{eq:dirDiff_A} at \( R = R_d \) demonstrates that \( R_d \) is a critical point of \( A \).
Next, the second derivative of \( A \) is computed as
\begin{align*}
    \dirDiff{\parenth{\dirDiff{A}{R}}}{R} &= \eta \cdot \frac{1}{2}\parenth{G R_d^T \delta R - \delta R^T R_d G}^\vee  \\
    &= \eta \cdot \frac{1}{2} \parenth{G R_d^T R \hat \eta + \hat \eta R^T R_d G}^\vee\\
    &= \eta \cdot \frac{1}{2} \bracket{\parenth{\braces{\tr{R^T R_d G} I - R^T R_d G} \eta}^\wedge}^\vee  \\
    &= \eta \cdot \frac{1}{2} \parenth{\tr{R^T R_d G} I - R^T R_d G}\eta ,
\end{align*}
where we used the scalar triple product rule, from~\cref{eqn:STP}, to arrive at the final form.
Evaluating the Hessian at \( R = R_d \)
\begin{align}\label{eq:A_hessian}
    \Hess(A)\cdot (\delta R,\delta R)\big|_{R=R_d} = \eta \cdot \frac{1}{2} \parenth{\tr{G}I -  G} \eta 
\end{align}
shows that the Hessian is positive definite at \( R_d \) and therefore the critical point is a minimum for any variation \( \eta \) and proves~\cref{item:prop_A_psd}.

\subsection{Proof of~\Cref{item:prop_A_critical_points}}\label{proof:prop_A_critical_points}
To prove~\cref{item:prop_A_critical_points} we must first determine the critical points of~\cref{eq:eRA}.
The derivations are originally presented in~\cite{bullo2004,chaturvedi2011a} and we present and expand on their original descriptions.
We define the relative attitude error as \( Q = R_d^T R \) and expand~\cref{eq:eRA} as
\begin{align*}
    e_{R_A} &= \frac{1}{2} \parenth{G Q - Q^T G} ^ \vee, \\
            &= \frac{1}{2} \parenth{%
            \begin{bmatrix} g_1 & 0 & 0 \\ 0 & g_2 & 0 \\ 0 & 0 & g_3 \end{bmatrix}
            \begin{bmatrix} q_{11} & q_{12} & q_{13} \\ q_{21} & q_{22} & q_{23} \\ q_{31} & q_{32} & q_{33}\end{bmatrix}
            -
            \begin{bmatrix} q_{11} & q_{21} & q_{31} \\ q_{12} & q_{22} & q_{32} \\ q_{13} & q_{23} & q_{33}\end{bmatrix}
            \begin{bmatrix} g_1 & 0 & 0 \\ 0 & g_2 & 0 \\ 0 & 0 & g_3 \end{bmatrix}
            }^\vee , \\
            &= \frac{1}{2} \begin{bmatrix} g_3 q_{32} - g_2 q_{23} \\ g_1 q_{13} - g_3 q_{31} \\ g_2 q_{21} - g_1 q_{12}\end{bmatrix}
\end{align*}
At a critical point, the following equations must be satisfied
\begin{subequations}\label{eq:erA_critical_conditons}
    \begin{align}
        g_3q_{32} &= g_2 q_{23}, \\
        g_1 q_{13} &= g_3 q_{31}, \\
        g_2 q_{21} &= g_1 q_{12}.
    \end{align}
\end{subequations}     
Using the conditions in~\cref{eq:erA_critical_conditons} the relative attitude error \( Q \) becomes
\begin{align}\label{eq:relative_attitude_error}
    Q = R_d^T R_e =
    \begin{bmatrix} 
        q_{11} & q_{12} & q_{13} \\ 
        \frac{g_1}{g_2} q_{12} & q_{22} & q_{23} \\
        \frac{g_1}{g_3} q_{13} & \frac{g_2}{g_3} q_{23} & q_{33}
    \end{bmatrix}
    \in \SO
\end{align}
Since, \( Q = R_d^T R \) is orthonormal the main diagonal of \( QQ^T \) is equal to \num{1} and each column of \( Q \) is unit length.
\begin{align*}
    Q Q^T &= I, \\
          &= \begin{bmatrix}
    Q_{11} & Q_{12} & Q_{13} \\ 
    Q_{21} & Q_{22} & Q_{23} \\ 
    Q_{31} & Q_{32} & Q_{33} 
\end{bmatrix}, 
\end{align*}
where
\begin{align*}
    Q_{11} &= q_{11}^2 + q_{12}^2 + q_{13}^2 ,\\
    Q_{12} &= q_{11}q_{12}\frac{g_1}{g_2} + q_{12}q_{22} + q_{13}q_{23} ,\\
    Q_{13} &= q_{11}q_{13}\frac{g_1}{g_3} + q_{12}q_{23}\frac{g_2}{g_3} + q_{13}q_{33},\\
    Q_{21} &= q_{11}q_{12}\frac{g_1}{g_2}  + q_{12}q_{22} + q_{23}q_{13} ,\\
    Q_{22} &= \frac{g_1^2}{g_2^2}q_{12}^2 + q_{22}^2 + q_{23}^2 ,\\
    Q_{23} &= \frac{g_1^2}{g_2g_3} q_{12}q_{13} + \frac{g_2}{g_3}q_{23}q_{22} + q_{23}q_{33},\\
    Q_{31} &= \frac{g_1}{g_3} q_{11}q_{13} + \frac{g_2}{g_3}q_{23}q_{12} + q_{33}q_{13},\\
    Q_{32} &= \frac{g_1^2}{g_3 g_2} q_{12}q_{13} + \frac{g_2}{g_3} q_{23}q_{22} + q_{33} q_{23},\\
    Q_{33} &= \frac{g_1^2}{g_3^2}q_{13}^2 + \frac{g_2^2}{g_3^2}q_{23}^2 + q_{33}^2.
\end{align*}
As a result the following \num{6} conditions must be satisfied when \( e_{R_A} = 0 \):
\begin{align}
    q_{11}^2 + q_{12}^2 + q_{13}^2 &= 1,\label{eq:orthogonal_1} \\
    \frac{g_1^2}{g_2^2}q_{12}^2 + q_{22}^2 + q_{23}^2 &= 1, \label{eq:orthogonal_2}\\
    \frac{g_1^2}{g_3^2}q_{13}^2 + \frac{g_2^2}{g_3^2}q_{23}^2 + q_{33}^2 &= 1, \label{eq:orthogonal_3}\\
    q_{11}^2 + \frac{g_1^2}{g_2^2}q_{12}^2 + \frac{g_1^2}{g_3^2}q_{13}^2 &= 1, \label{eq:unit_norm_1}\\
    q_{12} + q_{22}^2 + \frac{g_2^2}{g_3^2}q_{23}^2 &= 1, \label{eq:unit_norm_2}\\
    q_{13}^2 + q_{23}^2 + q_{33}^2 &= 1.\label{eq:unit_norm_3}
\end{align}
We can further simplify these relationships by rewriting~\cref{eq:orthogonal_1,eq:orthogonal_2,eq:orthogonal_3} as
\begin{align}
    q_{11}^2 &= 1 - q_{12}^2 - q_{13}^2 , \\
    q_{22}^2 &= 1 - \frac{g_1^2}{g_2^2} q_{12}^2 - q_{23}^2, \\
    q_{33}^2 &= 1 - \frac{g_1^2}{g_3^2}q_{13}^2 - \frac{g_2^2}{g_3^2}q_{23}^2 
\end{align}
and substituting these relationships into~\cref{eq:unit_norm_1,eq:unit_norm_2,eq:unit_norm_3} to arrive at
\begin{subequations}\label{eq:era_critical_point_conditions}
\begin{align}
    \parenth{1 - \frac{g_1^2}{g_2^2} } q_{12}^2 + \parenth{1 - \frac{g_1^2}{g_3^2}}q_{13}^2 &= 0 , \label{eq:era_condition_1}\\
    \parenth{1 - \frac{g_1^2}{g_2^2}} q_{12}^2 - \parenth{1 - \frac{g_2^2}{g_3^2}}q_{23}^2 &= 0 , \label{eq:era_condition_2}\\
    \parenth{1 - \frac{g_1^2}{g_3^2}} q_{13}^2 + \parenth{1 - \frac{g_2^2}{g_3^2}} q_{23}^2 &= 0.\label{eq:era_condition_3}
\end{align}
\end{subequations}     
The relationships in~\cref{eq:era_critical_point_conditions} are equivalent to the statement that \( e_{R_A} = 0 \) and must also be satisfied at each critical point.
Since each \( g_i \) is distinct and greater than zero, the solution to~\cref{eq:erA_critical_conditons}, or equivalently \( e_{R_A} = 0 \), must satisfy one of two possible conditions:
\begin{enumerate}[label=(\roman*)]
    \item \label{item:condition_1} \( q_{12} = q_{13} = q_{23} =0\)
    \item \label{item:condition_2} \( q_{ij} \neq 0 \; \forall \; i \neq j \)
\end{enumerate}
These conditions can be determined by noticing that~\cref{item:condition_1} is the trivial solution.
For~\cref{item:condition_2}, consider the situation that \( q_{12} \neq 0 \) then from the fact that \( g_i \) are distinct yields the fact that \( \parenth{1 - \frac{g_1^2}{g_2^2} } \neq 0 \) and \( \parenth{1 - \frac{g_1^2}{g_3^2} } \neq 0 \) which means that \( q_{13}^2 \neq 0 \) since the terms must add to zero.
Similar arguments can be made for the other relations in~\cref{eq:era_critical_point_conditions} which means that the solutions must satisify either~\cref{item:condition_1} or~\cref{item:condition_2}.

Now consider~\cref{item:condition_2} and assume that \( g_2 > g_1 \) and \( q_{12}, q_{13}, q_{23} \neq 0 \).
From~\cref{eq:era_condition_1} this implies that \( \parenth{1 - \frac{g_1^2}{g_2^2} } q_{12}^2 > 0\) since \( q_{12}^2 > 0\).
Therefore, in order for~\cref{eq:era_condition_1} to be satisfied means that \( \parenth{1 - \frac{g_1^2}{g_3^2} } q_{13}^2 < 0 \) which further implies that \( g_1 > g_3 \).
As a result, \( g_2 > g_1 > g_3 \).
Using this inequality in~\cref{eq:era_condition_3} implies that
\begin{align*}
    \parenth{1 - \frac{g_2^2}{g_3^2}} q_{23} < 0, \quad \parenth{1 - \frac{g_1^2}{g_3^2}} q_{13}^2 < 0.
\end{align*}
Therefore the left hand side of~\cref{eq:era_condition_3} is negative and a contradiction.
Applying a similar logic for the case \( g_1 > g_2 \) shows that~\cref{item:condition_2} is not possible and therefore~\cref{item:condition_1} must be satisfied.
Substituting~\cref{item:condition_1} into~\cref{eq:relative_attitude_error} shows that the main diagonal of \( Q \) must be composed of \( + 1 \) or \( -1 \).
Thus, the critical points of \( A(R) \) are those provided in~\cref{item:prop_A_critical_points}.
These critical points correspond to additional rotations of \SI{180}{\degree} about each axes, or more explicitly \( R_e = \braces{R_d, R_\alpha, R_\beta, R_\gamma} \) where
\begin{align}
    R_\alpha &= R_d \begin{bmatrix} 1 & 0 & 0 \\ 0 & -1 & 0 \\ 0 & 0 & -1 \end{bmatrix} , \\
    R_\beta &= R_d \begin{bmatrix} -1 & 0 & 0 \\ 0 & -1 & 0 \\ 0 & 0 & 1 \end{bmatrix} , \\
    R_\gamma &= R_d \begin{bmatrix} -1 & 0 & 0 \\ 0 & 1 & 0 \\ 0 & 0 & -1 \end{bmatrix}.
\end{align}

\subsection{Proof of~\Cref{item:prop_A_quadratic}}\label{proof:prop_A_quadratic}

The original proof for~\cref{item:prop_psi_quadratic} is presented in~\cite{lee2011a} and summarized in this section.
Define \( Q = R_d^T R = \exp \hat{\vc{x}} \in \SO \) for any \( \vc{x} \in \R^3 \) from Rodrigues' formula
\begin{align}\label{eq:rodrigues_formula}
    Q = \exp{\theta \hat{\vc{x}}} = I + \sin \theta \vh{x} + \parenth{1 - \cos \theta} \vh{x}^2 \in \SO.
\end{align}
Using a symbolic computation tool one can then find
\begin{align}\label{eq:psi_rodrigues}
    A &= \frac{1 - \cos \norm \vc{x}}{2\norm{\vc{x}^2}} \sum_{\parenth{i, j, k} \in \mathcal{C}} \parenth{g_i + g_j} x_k^2, \\
    \norm{e_{R_A}}^2 &= \frac{\parenth{1 - \cos \norm{\vc{x}}^2} }{4 \norm{\vc{x}}^4}  \sum_{\parenth{i, j, k} \in \mathcal{C}} \parenth{g_i - g_j}^2 x_i^2 x_j^2 + \frac{\sin^2 \norm{\vc{x}}}{4 \norm{\vc{x}}^2} \sum_{\parenth{i, j, k} \in \mathcal{C}} \parenth{g_i + g_j}^2 x_k^2,
\end{align}
where \( \mathcal{C} = \braces{ \parenth{1, 2, 3} , \parenth{2, 3, 1} , \parenth{3, 1, 2} } \).
Using~\cref{eq:psi_rodrigues} one can show
\begin{align}
    \frac{\norm{e_{R_A}}^2}{A} \leq \frac{h_2 + h_3}{h_1} , \quad \frac{A}{\norm{e_{R_A}}^2} \leq \frac{h_1 h_4}{h_5 \parenth{h_1 - \psi} } ,
\end{align}
which proves~\cref{item:prop_A_quadratic}.

\section{Proof of~\Cref{prop:attractive_error_dynamics}}\label{proof:attractive_error_dynamics}
From the kinematics \( \dot{R} = R \hat{\Omega}\), and noting that \( \dot{R}_d = 0 \) the time derivative of \( R_d^T R \) is given as
\begin{gather*}
	\diff{}{t} \parenth{R_d^T R} = R_d^T R \hat{e}_\Omega.
\end{gather*}
Applying this to the time derivative of~\cref{eq:A} gives
\begin{gather*}
	\diff{}{t} (A) = -\frac{1}{2} \tr{G R_d^T R \hat{e}_\Omega}.
\end{gather*}
Applying~\cref{eqn:hat1} into this shows~\cref{eq:A_dot}.

The time derivative of the attractive attitude error vector, \( e_{R_A} \), is given by
\begin{gather*}
	\diff{}{t} ( e_{R_A}) = \frac{1}{2} \parenth{\hat{e}_\Omega R^T R_d G + (R^T R_d G)^T \hat{e}_\Omega}^\vee.
\end{gather*}
Using the hat map property given in~\cref{eqn:xAAx} this is further reduced to~\cref{eq:eRA_dot,eq:E}.

% TODO Add link to dynamics that includes both external force and a disturbance
We show~\cref{eq:eW_dot} by rearranging~\cref{eq:Wdot} as 
\begin{align*}
    \diff{}{t} e_\Omega = \dot{\Omega} = J^{-1} \parenth{u - \Omega \times J \Omega + W(R,\Omega) \Delta +M_{ext}}.
\end{align*}

\section{Proof of~\Cref{prop:att_control}}\label{proof:att_control}
Consider the following Lyapunov function:
\begin{gather*}%\label{eqn:v_nodist}
	\mathcal{V} = \frac{1}{2} e_\Omega \cdot J e_\Omega + k_R \Psi(R,R_d). 
\end{gather*}
From~\cref{item:prop_psi_psd} of~\Cref{prop:attitude_control_configuration_error}, \(\mathcal{V} \geq 0 \).
Using~\cref{eq:eW_dot,eq:A_dot} with \( \Delta = 0 \), the time derivative of \( \mathcal{V} \) is given by
\begin{align*}%\label{eqn:vdot_nodist}
	\dot{\mathcal{V}} &= -k_\Omega \norm{e_\Omega}^2. 
%	e_\Omega^T \parenth{ u - \Omega \times J \Omega } + k_R e_R \cdot e_\Omega \nonumber \\
\end{align*}
Since \( \mathcal{V} \) is positive definite and \( \dot{\mathcal{V}} \) is negative semi-definite, the zero equilibrium point \( e_R, e_\Omega \) is stable in the sense of Lyapunov. 
This also implies \( \lim_{t\to\infty} \norm{e_\Omega} = 0 \) and \( \norm{e_{R_A}} \) is uniformly bounded, as the Lyapunov function is non-increasing. From~\cref{eq:eRA_dot}, $\lim_{t\to\infty} \dot e_R =0$. 
One can show that \( \norm{\ddot{e}_R} \) is bounded.
From Barbalat's Lemma, it follows \( \lim_{t\to\infty}\norm{\dot{e}_R} = 0 \)~\cite[Lemma 8.2]{khalil1996}. 
Therefore, the equilibrium is asymptotically stable. 
	
Furthermore, since \( \dot{\mathcal{V}} \leq 0 \) the Lyapunov function is uniformly bounded which implies 
\begin{align*}
	A(R(t)) \leq \mathcal{V}(t) \leq \mathcal{V}(0).
\end{align*}

\section{Proof of~\Cref{prop:repulsive_configuration_error}}\label{proof:repulsive_configuration_error}
To prove~\cref{item:prop_psi_psd} we note that~\cref{eq:A} is a positive definite function about \( R = R_d \)~\cite{bullo2004}, which is shown in~\cref{proof:prop_psi_psd}.
The constraint angle is assumed \( \ang{0} \leq \theta \leq \ang{90} \) such that \( 0 \leq \cos \theta \).
The term \( r^T R^T v \) represents the cosine of the angle between the body fixed vector \( r \) and the inertial vector \( v \). 
It follows that
\begin{align*}
	0 \leq  \frac{\cos \theta -  r^T R^T v}{1 + \cos \theta} \leq 1 ,
\end{align*}
for all \( R \in \SO \). 
As a result, its negative logarithm is always positive and proves that~\cref{eq:B} is positive definite.
The error function \( \Psi = A B \) is composed of two positive terms and is therefore also positive definite, and it is minimized at \( R = R_d \).

A straightforward application of the chain and product rules of differentiation allows us to show~\cref{item:prop_erb} as
\begin{align*}
	\dirDiff{B}{R} &=  \eta \cdot \frac{- \left( R^T v\right)^\vee r}{\alpha \left(\cos \theta - r^T R^T v \right)} ,
%	-\frac{1}{\alpha} \left[\left(\frac{1+\cos \theta}{\cos \theta - r^T R^T v} \right) \left(\eta \cdot \frac{\left( R^T v\right)^\vee r}{1+\cos \theta} \right) \right]
\end{align*}
where the scalar triple product~\cref{eqn:STP} was used.

We show~\cref{item:prop_psi_quadratic} by computing the hessian of \( \Psi \), namely \( \Hess(\Psi) \),  using the chain rule as 
\begin{align*}
    \Hess &(\Psi)\cdot (\delta R,\delta R)= (\mathbf{D}_R\parenth{\mathbf{D}_R A \cdot \delta R}\cdot \delta R) B\\
    & + \parenth{\dirDiff{A}{R} }\parenth{\dirDiff{B}{R}} 
    + \parenth{\dirDiff{B}{R} }\parenth{\dirDiff{A}{R}}\\ &+(\mathbf{D}_R\parenth{\mathbf{D}_R B \cdot \delta R}\cdot \delta R) A.
\end{align*}
The first order derivatives of \( A(R) \) and \( B(R) \) are given by~\cref{eq:eRA,eq:eRB}. 
The hessian of \( A(R) \) is shown in~\cref{eq:A_hessian}.
The hessian of \( B(R) \) is computed as
\begin{align*}
    \dirDiff{\parenth{\dirDiff{B}{R}}}{R} =& \eta \cdot \left[\frac{\parenth{\delta R^T v}^\wedge r}{\alpha\parenth{r^T R^T v - \cos \theta}} \right. \\
    & \left. - \frac{\parenth{R^T v}^\wedge r \parenth{r^T \delta R^T v}}{\alpha \parenth{r^TR^Tv - \cos \theta}^2}   \right].
\end{align*}
The term \( \parenth{\delta R^T v}^\wedge r\) is simplified as
\begin{align*}
    \parenth{\delta R^T v}^\wedge r &= - \hat{r} \delta R ^T v  \\
    &= - \hat{r} \parenth{R \hat \eta}^T v  \\
    &= \hat{r} \parenth{\hat{\eta} R^T } v  \\
    &= - \hat{r} \parenth{R^T v}^\wedge \eta ,
\end{align*}
where we utilized the hat map property from~\cref{eqn:cross_product}.
Similarly, the term \( r^T \delta R^T v \) is simplified as
\begin{align*}
    r^T \delta R^T v &= r^T \parenth{-\hat \eta R^T}v  \\
    &= r^T \parenth{R^T v}^\wedge \eta.
\end{align*}
The hessian of \( B(R) \) then becomes
\begin{align*}
    \dirDiff{\parenth{\dirDiff{B}{R}}}{R} =& \eta \cdot \left[ - \frac{\hat r \parenth{R^T v}^\wedge}{\alpha \parenth{r^T R^T v - \cos \theta}} \right. \\ 
    & \left. - \frac{\parenth{R^T v}^\wedge r r^T \parenth{R^T v}^\wedge }{\alpha\parenth{r^T R^T v - \cos \theta}^2}  \right] \eta.
\end{align*}
Using these terms, we evaluate \( \Hess{\Psi} \) at the desired attitude \( R = R_d \) as follows. 
Since $A=0$ and $\mathbf{D}_R A=0$ at $R=R_d$, 
\begin{align*}
    \Hess(\Psi)\cdot (\delta R,\delta R)\big|_{R=R_d} = \eta \cdot \frac{1}{2} B \parenth{\tr{G}I -  G} \eta , 
\end{align*}
which is positive definite since \( B > 1\) and \( \sum g_i > g_i\). 
The domain \( D \) is an open neighborhood of the desired attitude \( R_d \), and it excludes the undesired equilibrium points of \( A(R) \) and the infeasible regions defined by the constraints \( r^T R^T v_i \). 
Therefore, the only critical point of the error function $\Psi$ in the domain $D$ corresponds to the desired attitude $R=R_d$ with $e_R=0$ and $\Psi=0$. 
Therefore, in \( D \) the configuration error function is quadratic and the bounds in~\cref{item:prop_psi_quadratic} are valid according to~\cite[Proposition 6.30]{bullo2004}.

\section{Proof of~\Cref{prop:repulsive_error_dynamics}}\label{proof:repulsive_error_dynamics}
The time derivative of the repulsive error function is given by
\begin{gather*}
	\diff{}{t} (B) = \frac{r^T \parenth{\hat{\Omega} R^T} v}{\alpha \parenth{r^T R^T v - \cos \theta}}.
\end{gather*}
Using the scalar triple product, given by~\cref{eqn:STP}, one can reduce this to~\cref{eq:B_dot}.

We take the time derivative of the repulsive attitude error vector, \( e_{R_B} \), as
\begin{gather*}
	\diff{}{t}( e_{R_B} )= a \Omega v^T R r - a R^T v \Omega^T r + b R^T \hat{v} R r ,
\end{gather*}
with \( a \in \R \) and \( b \in \R\) given by 
\begin{gather*}
	a = \bracket{\alpha \parenth{r^T R^T v - \cos \theta}}^{-1} , \;
	b = \frac{r^T \hat{\Omega} R^T v}{\alpha \parenth{r^T R^T v - \cos \theta}^2}.
\end{gather*}
Using the scalar triple product from~\cref{eqn:STP} as \( r \cdot \Omega \times \parenth{R^T v} = \parenth{R^T v} \cdot r \times \Omega \) gives~\cref{eq:eRB_dot,eq:F}.

We show the time derivative of the configuration error function as
\begin{gather*}
	\diff{}{t} (\Psi) = \dot{A} B + A \dot{B}.
\end{gather*}
A straightforward substitution of~\cref{eq:A_dot,eq:B_dot,eq:A,eq:B} into this and applying~\cref{eq:eR} shows~\cref{eq:psi_dot}.

	
\section{Proof of~\Cref{prop:eR_dot_bound}}\label{proof:eR_dot_bound}

Consider the open neighborhood $D$ of $R=R_d$ defined in~\Cref{prop:repulsive_configuration_error}.
The proof of the upper bound of \( A(R) \) is given in~\cref{proof:prop_A_quadratic}.
The selected domain ensures that the configuration error function is bounded \( \Psi < \psi \).
This implies that that both \( A(R) \) and \( B(R) \) are bounded by constants \( c_A c_B < \psi < h_1\).
Furthermore, since \( \norm{B} > 1 \) this ensures that \( c_A, c_B < \psi\) and shows~\cref{eqn:AB_bound}.

Next, we show~\cref{eqn:E_bound,eqn:F_bound} using the Frobenius norm.
The Frobenius norm \( \norm{E}_F \) is given in~\cite{lee2013b} as
\begin{gather*}
	\norm{E}_F = \sqrt{\tr{E^T E}} = \frac{1}{2} \sqrt{\tr{G^2} + \tr{R^T R_d G}^2}.
\end{gather*}
Applying Rodrigues' formula and a symbolic computation tool, this is simplified to
\begin{gather*}
	\norm{E}^2_F \leq \frac{1}{4} \parenth{\tr{G^2} + \tr{G}^2} \leq \frac{1}{2} \tr{G}^2 ,
\end{gather*}
which shows~\cref{eqn:E_bound}, since \( \norm{E} \leq \norm{E}_F \).

To show~\cref{eqn:F_bound}, we apply the Frobenius norm \( \norm{F}_F \):
\begin{align*}
	\norm{F}_F =& \frac{1}{\alpha ^2 \parenth{r^T R^T v - \cos \theta}^2} \left[\tr{a^T a} - 2 \tr{a^T b} \right. \\
	&\left.+ 2 \tr{a^T c } + \tr{b^T b}  - 2 \tr{b^T c} + \tr{c^T c}\right].
\end{align*}
where the terms \( a, b, \text{ and } c \) are given by
\begin{gather*}
	a = r^T R r I , \quad	b = R^T v r^T , \quad c = \frac{R^T \hat{v} R r v^T R \hat{r}}{r^T R^T v - \cos \theta}.
\end{gather*}
A straightforward computation of \( a^T a \) shows that
\begin{gather*}
	\tr{a^T a} = \parenth{v^T R r}^2 \tr{I} \leq 3 \beta^2 ,
\end{gather*}
where we used the fact that \( v^T R r = r^T R^T v < \beta \) from our given domain.
Similarly, one can show that \( \tr{a^T b} \) is equivalent to
\begin{gather*}
	\tr{a^T b} = v^T R r \tr{R^T v r^T} = \parenth{v^T R r}^2 \leq \beta^2 ,
\end{gather*} 
where we used the fact that \( \tr{x y^T} = x^T y \).
The product \( \tr{a^T c} \) is given by
\begin{gather*}
	\tr{a^T c} = \frac{v^T R r}{r^T R^T v - \cos \theta} \tr{\parenth{R^T v}^\vee \parenth{r v^T R} \hat{r} } ,
\end{gather*}
where we used the hat map property~\cref{eqn:RxR}.
One can show that \(\mathrm{tr}[a^T c] \leq 0 \) over the range \( -1 \leq v^T R r \leq \cos \theta \). 
Next, \( \tr{b^T b}\) is equivalent to
\begin{gather*}
	\tr{b^T b} = \tr{r v^T R R^T v r^T} = 1 ,
\end{gather*}
since \( r,v \in \Sph^2\).
Finally, \( \tr{c^T c} \) is reduced to
\begin{gather*}
	\tr{c^T c} = \tr{\hat{r} R^T v r^T \bracket{-I + R^T v v^T R} r v^T R \hat{r}} ,
\end{gather*}
where we used the fact that \( \hat{x}^2 = - \norm{x}^2 I + x x^T\).
Expanding and collecting like terms gives
\begin{gather*}
	\tr{c^T c } = \frac{1 - 2\parenth{v^T R r}^2 + \parenth{v^T R r}^4}{\parenth{r^T R^T v - \cos \theta}^2}. 
\end{gather*}
Using the given domain \( r^T R^T v \leq \beta \) gives the upper bound~\cref{eqn:F_bound}.
The bound on \( e_{R_A} \) is given in~\cref{proof:prop_A_quadratic} while \( e_{R_B} \) arises from the definition of the cross product \( \norm{a \times b} = \norm{a} \norm{b} \sin \theta \).
Finally, we can find the upper bound~\cref{eq:eR_dot} as
\begin{gather*}
	\norm{\dot{e}_R} \leq \parenth{\norm{B} \norm{E} + 2 \norm{e_{R_A}} \norm{e_{R_B}} + \norm{A}\norm{F}} \norm{e_\Omega} \,.
\end{gather*}
Using~\cref{eqn:AB_bound} to \cref{eqn:eRB_bound} one can define \( H \) in terms of known values.

\section{Proof of~\Cref{prop:adaptive_control}}\label{proof:adaptive_control}
Consider the Lyapunov function \( \mathcal{V} \) given by
\begin{align*}
	\mathcal{V} = \frac{1}{2} e_\Omega \cdot J e_\Omega + k_R \Psi + c J e_\Omega \cdot e_R + \frac{1}{2 k_\Delta} e_\Delta \cdot e_\Delta , %\label{eqn:v_adapt}
\end{align*}
over the domain \( D \), defined in~\cref{prop:repulsive_configuration_error}. 
In this set, the properties of~\Cref{prop:repulsive_configuration_error,prop:repulsive_error_dynamics} are satisfied.
%Furthermore, the set \( D \) is an open neighborhood of the desired attitude and Lyapunov stability results are valid in this set.

From~\Cref{prop:repulsive_configuration_error}, the configuration error function is locally quadratic and it is bounded in \( D \) by~\cref{eq:psi_bound}.
Using this, the Lyapunov function \( \mathcal{V} \) is bounded by
\begin{align*} %\label{eqn:v_upper_bound}
	z^T W_1 z \leq \mathcal{V} \leq z^T W_2 z ,
\end{align*}
where \( e_\Delta = \Delta - \bar{\Delta} \), \( z = [\|e_R\|,\|e_\Omega\|,\|e_\Delta\|]^T\in\R^3 \) and the matrices \(W_1,W_2 \in \R^{3 \times 3}\) are given by
\begin{align*}
	W_1  = \begin{bmatrix}
		k_R n_1 & -\frac{1}{2} c \lambda_M & 0 \\
		-\frac{1}{2} c \lambda_M & \frac{1}{2} \lambda_m & 0 \\
		0 & 0 & \frac{1}{2 k_\Delta}
	\end{bmatrix}\quad
	W_2  = \begin{bmatrix}
		k_R n_2 & \frac{1}{2} c \lambda_M & 0 \\
		\frac{1}{2} c \lambda_M & \frac{1}{2} \lambda_M & 0 \\
		0 & 0 & \frac{1}{2 k_\Delta}
	\end{bmatrix}.
\end{align*}
%\begin{gather*}
%\frac{1}{2}\lambda_m k_R n_1 - \frac{1}{4}c^2 \lambda_M^2 >0\\
%\frac{2\lambda_m k_R n_1}{\lambda_M^2} > c^2 \\
%\end{gather*}
The time derivative of \( \mathcal{V}\) with the control input defined by~\cref{eqn:adaptive_control} is given as
\begin{align*}
	\dot{\mathcal{V}} =& - k_\Omega e_\Omega^T e_\Omega + \parenth{e_\Omega + c e_R}^T W e_\Delta - k_R c e_R^T e_R \nonumber\\
	&- k_\Omega c e_R^T e_\Omega + c J e_\Omega^T \dot{e}_R - \frac{1}{k_\Delta} e_\Delta^T \dot{\bar{\Delta}} , \label{eqn:vdot}
\end{align*}
where we used \( \dot{e}_\Delta = - \dot{\bar{\Delta}} \).
The terms linearly dependent on \( e_\Delta\) are combined with~\cref{eqn:delta_dot} to yield the following
\begin{align*}
	 e_\Delta^T \parenth{W^T \parenth{e_\Omega + c e_R} - \frac{1}{k_\Delta} \dot{\bar{\Delta}}} = 0. 
\end{align*}
Using~\Cref{prop:eR_dot_bound}, an upper bound on \( \dot{\mathcal{V}} \) is written as
\begin{gather*}
	\dot{\mathcal{V}} \leq -\zeta^T M \zeta ,
\end{gather*}
where $\zeta=[\|e_R\|,\|e_\Omega\|]\in\R^2$, and the matrix \( M \in \R^{2 \times 2} \) is 
\begin{gather*}
	M = \begin{bmatrix}
		k_R c & \frac{k_\Omega c}{2} \\
		\frac{k_\Omega c}{2} & k_\Omega - c \lambda_M H
	\end{bmatrix}.
\end{gather*}
%\begin{gather*}
%k_R c (k_\Omega -c \lambda_M H) -\frac{1}{4}k_\Omega^2 c^2 < 0\\
%k_R (k_\Omega -c \lambda_M H) -\frac{1}{4}k_\Omega^2 c < 0\\
%4k_R k_\Omega -c(4k_R \lambda_M H +k_\Omega^2) <0
%\end{gather*}

If \( c \) is chosen such that~\cref{eqn:c_bound} is satisfied then the matrices \( W_1, W_2 \) and \( M \) are positive definite.
This implies that $\mathcal{V}$ is positive definite and decrescent, and $\dot{\mathcal{V}}$ is negative semidefinite in the domain $D$. As such, the zero equilibrium is stable in the sense of Lyapunov, and all of the error variables are bounded. Furthermore, $\lim_{t\to\infty} \zeta=0$ according to the LaSalle-Yoshizawa theorem~\cite{khalil1996}. 

%and 
%As the Lyapunov function is non-increasing $z$ is uniformly bounded. 
%	
%	
%	This implies that the Lyapunov function \( V \) is non-increasing and bounded from below.
%	Furthermore, since \( V \) is nonincreasing the disturbance error \( e_\Delta\) is bounded.
%	Since \( \dot{V} \) is uniformly continuous in time this implies that the zero equilibrium \( e_R, e_\Omega\) is stable in the sense of Lyapunov
%	The domain \( D \) does that include the undesired equilibria and ensures that as \( e_R \to 0 \) the attitude also converges \( R \to R_d \).
%%%%%%%%%%%%%%%%%%%%%%%%%%%%%%%%%%%%%%%%%%%%%%%%%%%%%%%%%%%%%%%%%%%%%%%%%%%%%%%%

\section{Proof of~\Cref{prop:translational_control}}\label{proof:translational_control}
The translational dynamics can be defined as
\begin{align}\label{eq:translation_dynamics_simple}
    \dot{x} &= v , \\
    m \dot{v} &= F_{ext} + u_f ,
\end{align}
where \( m \in \R \) is defined as the total mass of the vehicle while \( F_{ext} \) is the total external force.

Consider the Lyapunov function
\begin{align}\label{eq:translational_lyapunov}
    V = \frac{1}{2} k_x e_x^T e_x + \frac{1}{2} m e_v^T e_v + c e_x^T e_x.
\end{align}
\Cref{eq:translational_lyapunov} is also equivalent to
\begin{align}\label{eq:translational_lyapunov_matrix}
    V = z^T W_1 z , 
\end{align}
where \( z = \begin{bmatrix} \norm{e_x} & \norm{e_v} \end{bmatrix}^T \) and \( W_1\in \R^2\) is defined as
\begin{align}\label{eq:translational_lyapunov_W}
    W_1 = \begin{bmatrix}
        \frac{1}{2} k_x & \frac{1}{2} c \\
        \frac{1}{2} c & \frac{1}{2} m 
    \end{bmatrix}.
\end{align}
The time derivative of~\cref{eq:translational_lyapunov} can then be defined as
\begin{align}\label{eq:translation_lyapunov_derivative}
    \dot{\mathcal{V} } = - \parenth{k_v - c} e_v^T e_v - \frac{k_x c}{m} e_x^T e_x - \frac{k_v c}{m} e_x^T e_v, 
\end{align}
where we used~\cref{eq:translational_control,eq:translation_error_variables,eq:translation_error_dynamics}.
\Cref{eq:translation_lyapunov_derivative} can also be redefined as
\begin{align}\label{eq:translational_lyapunov_derivative_matrix}
    \dot{\mathcal{V}} = - z^T W_2 z, 
\end{align}
where \( W_2 \in \R^2 \) is defined as
\begin{align}\label{eq:translational_lyapunov_W2}
    W_2 = \begin{bmatrix}
        \frac{k_x c}{m} & \frac{k_v^2 c^2}{2 m^2} \\
        \frac{k_v c}{2 m} & k_v - c
    \end{bmatrix}.
\end{align}
If \( c \) is chosen such that~\cref{eq:translational_control_gain_bound} is satisfied then \( W_1, W_2 \) will be positive definite.
As a result, this shows that \( \mathcal{V} > 0 , \dot{ \mathcal{V}} < 0 \) and that the zero equilibrium of the error variables, namely \( x_d, \dot{x}_d \) is asymptotically stable~\cite{khalil1996}.



