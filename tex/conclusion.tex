% !TEX root = ../dissertation.tex

\chapter{Conclusion}\label{sec:conclusion}
Summarize the paper. 
Highlight all the contributions.

Conclusion
\begin{itemize}
    \item What are the lessons learned?
    \item What were the overall insights?
    \item Was the problem completely solved? 
    \item Don't copy the introduction
\end{itemize}

\section{Future Work and Conclusions}

The developments in this dissertation we developed with the assumption the location of the spacecraft and the motion of the small body are both accurately known.
In reality this is far from the norm and both are unknown and subject of stochastic errors from uncertain measurements and dynamics. 
Developing mapping and control schemes that probablisiticly estimate both the spacecraft and small body motion would be a worthwhile research topic.
Furthmore, this approach could leverage some of the recent advances in \gls{slam} and extend their use to the space domain.
A number of recent publications, such as \textcite{cocaud2010,cocaud2012,vassallo2015}, have investigated the use of \gls{slam} techniques around asteroids.
One key issue for the space domain is that in typical \gls{slam} approaches, such as that for autonomous driving, the terrain is assumed to be fixed.

Extending these methods to consider uncertainty in both measurement and dyanmics would greatly increase the applicability of the approaches presented in this dissertation.
Careful consideration should be placed on numerical techniques which respect the geometric properties of the dynamic environments.
Recent work, such as~\textcite{kulumani2017}, have investigated Bayesian estimation schemes on \( \SO \). 
Extending their use to the full space of rigid body motions would enable a holistic approach to the solution of the full two body problem.


